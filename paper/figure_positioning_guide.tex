% LaTeX Figure Positioning Guide to Prevent Blank Spaces
% =====================================================

% 1. FIGURE PLACEMENT OPTIONS
% ---------------------------
% Replace [H] with [!htbp] for better flexibility:
% ! = Override LaTeX's internal parameters
% h = Try to place here (at the current position)
% t = Try to place at top of page
% b = Try to place at bottom of page  
% p = Try to place on a separate page for floats

% AVOID: [H] - Forces exact placement but can create large blank spaces
% USE: [!htbp] - Gives LaTeX flexibility to find the best position

% 2. ADDITIONAL PACKAGES TO ADD TO PREAMBLE
% ------------------------------------------
\usepackage{placeins}  % Provides \FloatBarrier command
\usepackage{afterpage} % Provides \afterpage command for better control

% 3. SPACING ADJUSTMENTS
% ----------------------
% Reduce excessive vertical spacing in multi-part figures:
\vspace{0.3cm}  % Instead of \vspace{0.5cm} or larger

% Control spacing around figures:
\setlength{\textfloatsep}{10pt plus 2pt minus 2pt}     % Space between floats and text
\setlength{\floatsep}{8pt plus 2pt minus 2pt}          % Space between floats
\setlength{\intextsep}{10pt plus 2pt minus 2pt}        % Space around in-text floats

% 4. FLOAT BARRIER USAGE
% ----------------------
% Use \FloatBarrier before important section breaks to prevent figures from floating too far:
\FloatBarrier
\section{New Section}

% 5. ALTERNATIVE APPROACHES FOR PROBLEMATIC FIGURES
% -------------------------------------------------

% Option A: Use \afterpage for large figures
\afterpage{
    \begin{figure}[!htbp]
        \centering
        \includegraphics[width=\textwidth]{figure.png}
        \caption{Caption text}
        \label{fig:label}
    \end{figure}
}

% Option B: Use minipage for better control of multi-part figures
\begin{figure}[!htbp]
    \centering
    \begin{minipage}{\textwidth}
        \centering
        \begin{minipage}[t]{0.48\textwidth}
            \centering
            \includegraphics[width=\textwidth]{fig1.png}
            \subcaption{Caption 1}
        \end{minipage}
        \hfill
        \begin{minipage}[t]{0.48\textwidth}
            \centering
            \includegraphics[width=\textwidth]{fig2.png}
            \subcaption{Caption 2}
        \end{minipage}
    \end{minipage}
    \caption{Main caption}
\end{figure}

% Option C: For very large figures, consider using landscape orientation
\usepackage{pdflscape}
\begin{landscape}
    \begin{figure}[!htbp]
        \centering
        \includegraphics[width=\textwidth]{large_figure.png}
        \caption{Large figure caption}
    \end{figure}
\end{landscape}

% 6. DEBUGGING BLANK SPACES
% -------------------------
% Add this to preamble to see float placement in draft mode:
\usepackage[draft]{graphicx}  % Shows boxes instead of images for faster compilation
\usepackage{showframe}        % Shows page margins and text blocks

% 7. BEST PRACTICES SUMMARY
% -------------------------
% - Always use [!htbp] instead of [H]
% - Add \FloatBarrier before major section breaks
% - Reduce excessive \vspace in multi-part figures
% - Consider text flow when placing multiple figures
% - Use \afterpage for figures that must appear after current page
% - Test compilation frequently to catch spacing issues early
% - Consider breaking very large multi-part figures into separate figures

% 8. EMERGENCY FIXES
% ------------------
% If you still get blank spaces, try these in order:

% Fix 1: Add \raggedbottom to preamble (allows variable page heights)
\raggedbottom

% Fix 2: Increase float placement parameters
\renewcommand{\topfraction}{0.9}        % Max fraction of page for top floats
\renewcommand{\bottomfraction}{0.8}     % Max fraction of page for bottom floats  
\renewcommand{\textfraction}{0.1}       % Min fraction of page for text
\renewcommand{\floatpagefraction}{0.8}  % Min fraction of float page that must be floats

% Fix 3: For desperate cases, use \clearpage to force page break
\clearpage  % Forces all pending floats to be placed before continuing 