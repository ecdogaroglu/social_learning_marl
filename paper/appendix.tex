\setcounter{proposition}{0}
\setcounter{assumption}{0}
\setcounter{lemma}{0}
\setcounter{theorem}{0}

\chapter{Reinforcement Learning Background}

\label{appendix:rl_details}
This section provides a brief overview of reinforcement learning (RL) and multi-agent reinforcement learning (MARL) to provide a foundation for understanding the theoretical framework presented in this paper.

\section{Reinforcement Learning Foundations}

Reinforcement learning (RL) provides a mathematical framework for solving sequential decision-making problems under uncertainty. In the classical single-agent setting, an agent interacts with a stationary environment by observing states, taking actions, and receiving rewards, with the objective of maximizing its expected cumulative reward over time \cite{sutton2018reinforcement}. This interaction is formalized as a Markov Decision Process (MDP), defined as a tuple $M = \langle S, A, T, R, \gamma \rangle$ where:

\begin{itemize}
    \item $S$ is the state space, representing all possible configurations of the environment
    \item $A$ is the action space, representing all possible decisions available to the agent
    \item $T: S \times A \mapsto \Delta(S)$ is the transition function, specifying the probability distribution over next states given the current state and action
    \item $R: S \times A \mapsto \mathbb{R}$ is the reward function, specifying the immediate reward received after taking an action in a state
    \item $\gamma \in [0, 1)$ is the discount factor, balancing immediate versus future rewards
\end{itemize}

The agent's behavior is characterized by a policy $\pi: S \mapsto \Delta(A)$, which maps states to probability distributions over actions. The policy can be evaluated using value functions: the state-value function $V^\pi(s)$ represents the expected return when starting in state $s$ and following policy $\pi$ thereafter, while the action-value function $Q^\pi(s, a)$ represents the expected return after taking action $a$ in state $s$ and following policy $\pi$ thereafter. The goal of RL is to find an optimal policy $\pi^*$ that maximizes the expected return from all states.

The Markov property, which states that the future is independent of the past given the present, is a fundamental assumption in MDPs. This property ensures that the current state provides all the necessary information for making optimal decisions, greatly simplifying the learning problem. However, this assumption becomes problematic in multi-agent settings, as we will discuss next.

\section{Multi-Agent Reinforcement Learning: Concepts and Challenges}

Multi-agent reinforcement learning (MARL) extends the single-agent RL framework to environments with multiple autonomous agents that interact simultaneously \cite{marl-book,yang2020overview, huh2024multiagentreinforcementlearningcomprehensive, busoniu2010multi, zhang2021multi}. MARL encompasses a wide spectrum of scenarios, from fully cooperative tasks where agents share a common reward function, to fully competitive zero-sum games, to the general mixed cooperative-competitive case with individual reward functions \cite{nowé2012game}. These interactions are commonly formalized as Markov games (also known as stochastic games) \cite{littman1994markov, shapley1953stochastic}, defined as a tuple $M_n = \langle I, S, A, T, R, \gamma \rangle$, where:

\begin{itemize}
    \item $I = \{1, \ldots, n\}$ is the set of $n$ agents
    \item $S$ is the state space, shared among all agents
    \item $A = \times_{i \in I} A^i$ is the joint action space, where $A^i$ is the action space of agent $i$
    \item $T: S \times A \mapsto \Delta(S)$ is the transition function that depends on the joint action
    \item $R = \times_{i \in I} R^i$ is the joint reward function, where $R^i: S \times A \mapsto \mathbb{R}$ is agent $i$'s individual reward function
    \item $\gamma \in [0, 1)$ is the discount factor
\end{itemize}
MARL introduces several fundamental challenges beyond those encountered in single-agent RL. The joint action space grows exponentially with the number of agents, creating a combinatorial explosion that makes exploration and learning increasingly difficult as more agents are added to the system. Coordination challenges further complicate the learning process, as agents must synchronize their actions to achieve effective joint behavior, especially in cooperative settings where team performance depends on complementary actions \cite{du2023reviewcooperationmultiagentlearning}, that can also involve explicit communication \cite{zhu2024surveymultiagentdeepreinforcement}. In scenarios with shared rewards, the credit assignment problem becomes particularly troublesome, as determining individual contributions to team success grows increasingly complex when outcomes result from joint actions rather than individual decisions. MARL also demands sophisticated strategic reasoning, requiring agents to model and reason about other agents' goals, beliefs, and strategies, particularly in competitive or mixed cooperative-competitive settings. Perhaps most critically, when multiple agents learn simultaneously, the environment becomes non-stationary from each agent's perspective, as other agents' changing policies continuously modify the effective environment dynamics, violating a fundamental assumption of traditional reinforcement learning algorithms.

\section{The Non-Stationarity Challenge}
The non-stationarity problem in MARL represents a fundamental departure from single-agent RL assumptions and presents one of the most significant obstacles to effective multi-agent learning \cite{hernandez2017survey, papoudakis2019dealing}. To understand this challenge more precisely, we can analyze how learning dynamics alter the effective environment for each agent. When multiple agents learn simultaneously, each agent $i$ with policy $\pi^i$ effectively faces an environment whose dynamics depend on the joint policies of all other agents $\pi^{-i}$. From agent $i$'s perspective, the effective transition function becomes:
\begin{equation}
    T^i_{\boldsymbol{\pi}^{-i}_t}(s_{t+1}|s_t, a^i_t) = \sum_{\mathbf{a}^{-i} \in A^{-i}} \left(\prod_{j \in I \setminus \{i\}} \pi^j_t(a^j|s_t)\right) \cdot T(s_{t+1}|s_t, (a^i_t, \mathbf{a}^{-i}))
\end{equation}
where $A^{-i}$ denotes the joint action space, $\mathbf{a}^{-i}$ denotes the joint action and $\boldsymbol{\pi}^{-i}_t$ denotes the joint policies of all agents except $i$. When other agents update their policies from $\boldsymbol{\pi}^{-i}_t$ to $\boldsymbol{\pi}^{-i}_{t+1}$ through learning, this effective transition function changes, creating a non-stationary environment for agent $i$. This non-stationarity violates the Markov property assumption underlying most RL algorithms and can lead to several significant challenges that fundamentally undermine the theoretical foundations of single-agent RL. When multiple agents learn simultaneously, standard RL convergence guarantees no longer apply, and learning algorithms may oscillate or diverge as the effective environment continuously shifts \cite{bowling2002multiagent}. This dynamic environment causes value function approximations to become increasingly inaccurate over time, leading to suboptimal policy decisions as agents base their choices on outdated models of their environment \cite{lauer2000algorithm}. Further complicating matters, agents face a moving target problem where they must simultaneously learn optimal policies while adapting to the evolving strategies of others, creating a complex coupled learning process that resists straightforward optimization approaches \cite{laurent2011world}. The situation becomes particularly problematic in competitive settings, where learning dynamics may lead to cyclic policy changes rather than convergence to stable strategies, as agents continuously adapt and counter-adapt to each other's evolving behaviors \cite{balduzzi2018mechanics}. These interconnected challenges highlight why addressing non-stationarity remains one of the central research questions in multi-agent reinforcement learning.

\section{Traditional Approaches to Non-Stationarity}
Researchers have developed various approaches to address the non-stationarity challenge in MARL. These can be broadly categorized as follows:

\subsection{Independent Learning}
The simplest approach is to ignore non-stationarity entirely, treating other agents as part of the environment and applying standard single-agent RL algorithms independently for each agent \cite{tan1993multi}. This approach, often called Independent Learning (IL), requires no explicit coordination or modeling of other agents. Methods in this category include Independent Q-Learning \cite{tampuu2017multiagent}, where each agent maintains its own Q-function and updates it using only its own experiences. While computationally efficient and naturally scalable to many agents, independent learning lacks theoretical convergence guarantees and can fail in complex multi-agent scenarios due to the violation of the stationarity assumption.

\subsection{Centralized Training with Decentralized Execution}
To mitigate non-stationarity during learning while preserving autonomous execution, many approaches adopt the paradigm of centralized training with decentralized execution (CTDE) \cite{lowe2017multi, foerster2018counterfactual, sunehag2018value}. In CTDE, agents have access to additional information during training (e.g., joint actions, global state, other agents' policies) but operate based solely on their local observations during execution.

Notable algorithms in this category include Multi-Agent Deep Deterministic Policy Gradient (MADDPG) \cite{lowe2017multi}, which uses centralized critics with access to joint actions and states during training but decentralized actors during execution. Another significant approach is Counterfactual Multi-Agent Policy Gradients (COMA) \cite{foerster2018counterfactual}, which addresses credit assignment using a centralized critic that computes a counterfactual baseline. The category also features QMIX \cite{rashid2018qmix}, which learns a centralized value function that factorizes into a mixing of individual agent Q-values, ensuring consistency between centralized and decentralized policies. While CTDE methods help stabilize learning, they still do not fully address the fundamental non-stationarity issue, as they focus on adapting to current policies rather than reasoning about policy evolution over time.

\subsection{Opponent Modeling and Population-Based Training}
Another approach is to explicitly model the behavior and learning processes of other agents, enabling more informed adaptation \cite{albrecht2018autonomous, he2016opponent}. This can involve predicting other agents' policies, types, or learning dynamics. Methods in this category include Deep Reinforcement Opponent Network (DRON) \cite{he2016opponent}, which integrates opponent modeling into deep Q-learning. Another significant approach is Probabilistic Recursive Reasoning (PR2) \cite{wen2019probabilistic}, which models higher-order beliefs about other agents' reasoning processes.

The category also encompasses Population-Based Training (PBT) \cite{jaderberg2019human}, which trains a population of agents simultaneously, creating a naturally changing learning environment that promotes robustness to non-stationarity. While these approaches can better handle changing policies, they typically focus on adaptation to current policies rather than influencing the learning trajectories of other agents.

\subsection{Equilibrium Learning and Stability Concepts}
Drawing on game theory, some approaches focus on finding policies that constitute game-theoretic solution concepts like Nash equilibria \cite{bowling2005convergence, hu2003nash, littman2001friend}. These methods aim to find stable joint policies where no agent has an incentive to unilaterally deviate. Notable algorithms include Nash Q-Learning \cite{hu2003nash}, which converges to Nash equilibria in general-sum Markov games. Another significant approach is Friend-or-Foe Q-Learning \cite{littman2001friend}, which converges to optimal policies in restricted classes of Markov games.

The category also features WoLF-PHC (Win or Learn Fast - Policy Hill Climbing) \cite{bowling2002multiagent}, which adjusts learning rates based on whether an agent is "winning" or "losing" to promote convergence. While these approaches provide theoretical guarantees under certain conditions, they often make strong assumptions about the game structure and other agents' rationality. Moreover, they focus on convergence to static equilibria rather than the dynamic nature of multi-agent learning.

\subsection{Meta-Learning for Non-Stationarity}
Recent work has explored meta-learning as a framework for rapid adaptation to non-stationarity in MARL \cite{alshedivat2018continuous, kim2021policy}. These approaches train agents to quickly adapt to new agents or environments, treating non-stationarity as a meta-learning problem. Key methods include Continuous Adaptation via Meta-Learning (CAML) \cite{alshedivat2018continuous}, which uses meta-learning to quickly adapt to evolving opponents.

Another significant approach is Meta-MAPG (Meta-Multiagent Policy Gradient) \cite{kim2021policy}, which extends meta-learning to explicitly account for the influence of an agent's actions on the learning processes of other agents. While meta-learning approaches show promise for rapid adaptation, they often still lack a comprehensive framework for modeling and influencing the long-term learning dynamics of multi-agent systems.

\section{Active Markov Games}

Active Markov Games provide a more sophisticated framework for modeling the dynamic nature of multi-agent learning by explicitly incorporating the policy update processes of all agents \cite{kim2022influencing}. They extend standard Markov games to capture not just the immediate effects of actions on states and rewards, but also their influence on future policy updates, addressing the non-stationarity challenge at a fundamental level.

\subsection{Formal Definition}

An Active Markov Game is defined as a tuple $M_n = \langle I, S, A, T, R, \Theta, U \rangle$, where:

\begin{itemize}
    \item $I = \{1, \ldots, n\}$ is the set of $n$ agents
    \item $S$ is the state space
    \item $A = \times_{i \in I} A^i$ is the joint action space, where $A^i$ is the action space of agent $i$
    \item $T: S \times A \mapsto \Delta(S)$ is the transition function
    \item $R = \times_{i \in I} R^i$ is the joint reward function, where $R^i: S \times A \mapsto \mathbb{R}$ is agent $i$'s individual reward function
    \item $\Theta = \times_{i \in I} \Theta^i$ is the joint policy parameter space, where $\Theta^i$ is the policy parameter space for agent $i$
    \item $U = \times_{i \in I} U^i$ is the joint Markovian policy update function, where $U^i: \Theta^i \times T^i \mapsto \Delta(\Theta^i)$ maps current policy parameters and transitions to distributions over next policy parameters
\end{itemize}
Here, $T^i$ represents the trajectory space for agent $i$, with a particular element $\tau^i_t = \{s_t, \mathbf{a}_t, r^i_t, s_{t+1}\}$ consisting of the current state, joint action, reward received, and the next state. The interaction process in an Active Markov Game unfolds through a sequential procedure of actions and updates. At timestep $t$, each agent $i$ selects an action $a^i_t \sim \pi^i(\cdot|s_t; \theta^i_t)$ based on its parameterized policy with parameters $\theta^i_t \in \Theta^i$. Following this selection, the environment transitions from state $s_t$ to $s_{t+1}$ according to $T(s_{t+1}|s_t, \mathbf{a}_t)$ based on the joint action $\mathbf{a}_t = (a^1_t, \ldots, a^n_t)$. Each agent $i$ then receives a reward $r^i_t = R^i(s_t, \mathbf{a}_t)$ and subsequently updates its policy parameters according to its update function $\theta^i_{t+1} \sim U^i(\theta^i_{t+1}|\theta^i_t, \tau^i_t)$. This process continues until non-stationary policies converge to a recurrent set of joint policies. The key insight is that agent $i$'s actions not only affect immediate states and rewards but also influence future policy updates of all agents, including itself, through the trajectory $\tau^i_t$.

\subsection{Augmented Transition Function and Stationarity}

The Active Markov Game framework fundamentally transforms the non-stationarity problem by explicitly modeling how policies evolve through Markovian update functions. To understand this transformation, let us contrast the transition functions in standard MARL and Active Markov Games. In standard MARL, the transition function $T: S \times A \mapsto \Delta(S)$ only captures state transitions based on joint actions. However, Active Markov Games introduce a critical innovation: they explicitly model how policies change via Markovian update functions. The augmented transition function becomes:

\begin{equation}
    \mathbb{P}(s_{t+1}, \boldsymbol{\theta}_{t+1}|s_t, \boldsymbol{\theta}_t) = \sum_{\mathbf{a}_t \in A} \left(\prod_{i \in I} \pi^i(a^i_t|s_t; \theta^i_t)\right) \times T(s_{t+1}|s_t, \mathbf{a}_t) \times \boldsymbol{U}(\boldsymbol{\theta}_{t+1}|\boldsymbol{\theta}_t, \boldsymbol{\tau}_t)
\end{equation}
The crucial difference is the inclusion of $\boldsymbol{U}$ - a Markovian update function that specifies exactly how policy parameters $\boldsymbol{\theta}_t$ (and consequently policies $\boldsymbol{\pi}$) evolve based on current parameters and trajectories. By making policy updates an explicit part of the system dynamics, what was previously an unpredictable external process becomes a predictable internal one. This augmented transition function operates over the joint space $(s, \boldsymbol{\theta}) \in S \times \Theta$, creating a joint process that is stationary even though individual policies are changing.


\subsection{Theoretical Properties and Convergence}

Active Markov Games exhibit important theoretical properties that enable rigorous analysis of multi-agent learning dynamics:

\paragraph{Stationary Periodic Distributions.} Under mild assumptions, the joint process of states and policies in an Active Markov Game converges to a stationary periodic distribution \cite{kim2022influencing}:

\begin{equation}
    \mu_k(s, \theta|s_0, \theta_0, \ell) = \mathbb{P}(s_t = s, \theta_t = \theta|s_0, \theta_0, \ell)
\end{equation}

where $\ell = t\%k$ with $\%$ denoting the modulo operation. This distribution represents the limiting behavior of the system after convergence, characterized by potentially periodic patterns of states and policies. The parameter $k$ represents the period length, with $k=1$ corresponding to fixed-point convergence. To address sensitivity to initial conditions, the framework introduces the concept of \textit{stochastically stable distributions} \cite{kim2022influencing}. These are limiting distributions that emerge when small random perturbations are added to policy updates, providing robustness to initial state and policy configurations.

\paragraph{Long-term Optimization Objective.} In Active Markov Games, agents optimize for long-term average reward rather than discounted return \cite{sutton2018reinforcement, kim2022influencing}:

\begin{equation}
    \max_{\theta^i, U^i} \rho^i(s, \theta, U) := \max_{\theta^i, U^i} \lim_{T \to \infty} \mathbb{E}\left[\frac{1}{T}\sum_{t=0}^T R^i(s_t, \mathbf{a}_t) \Bigg|
        \begin{array}{c}
            s_0=s, \theta_0=\theta,                                              \\
            \mathbf{a}_{0:T} \sim \boldsymbol{\pi}(\cdot|s_{0:T}; \theta_{0:T}), \\
            s_{t+1} \sim T(\cdot|s_t, \mathbf{a}_t),
            \theta_{t+1} \sim U(\cdot|\theta_t, \boldsymbol{\tau}_t)
        \end{array}
        \right]
\end{equation}

where  $0:T$ denotes the sequence from time $0$ to $T$. This formulation encourages agents to consider how to influence the limiting behavior of the system rather than just short-term performance, addressing the non-stationarity challenge at a fundamental level.

\paragraph{Active Equilibrium.} Active Markov Games give rise to a new solution concept called the Active Equilibrium \cite{kim2022influencing}, which generalizes traditional game-theoretic equilibria like Nash equilibrium. An Active Equilibrium is a joint policy parameter $\theta^* = \{\theta^{i*}, \theta^{-i*}\}$ with associated joint update function $U^* = \{U^{i*}, U^{-i*}\}$ such that:

\begin{equation}
    \rho^i(s, \theta^{i*}, \theta^{-i*}, U^{i*}, U^{-i*}) \geq \rho^i(s, \theta^i, \theta^{-i*}, U^i, U^{-i*})
\end{equation}
for all $i \in I$, $s \in S$, $\theta^i \in \Theta^i$, $U^i \in U^i$. This equilibrium concept captures the idea that rational agents should optimize not just their immediate policies but also their adaptation strategies, taking into account the learning dynamics of the system. The active equilibrium generalizes several classic solution concepts in game theory. Stationary Nash equilibria and correlated equilibria are special cases of active equilibria when $k = 1$ (fixed-point convergence) and joint action distributions are independent or correlated, respectively. Similarly, cyclic Nash equilibria and cyclic correlated equilibria can be viewed as special cases of active equilibria when $k > 1$ (periodic behavior), the joint update function is deterministic, and joint action distributions are independent or correlated, respectively. This generality allows the active equilibrium concept to capture a wider range of stable multi-agent behaviors in dynamic learning settings than traditional equilibrium notions.


Active Markov Games address the non-stationarity challenge in several fundamental ways. By incorporating policy parameters and update functions directly into the framework, Active Markov Games make the non-stationarity explicit and amenable to analysis, embracing it as an integral part of the multi-agent learning process rather than treating it as an external challenge to be mitigated. The average reward objective promotes farsighted optimization by encouraging agents to consider the long-term limiting behavior of the system rather than myopically optimizing for immediate or short-term rewards, leading to policies that shape the learning trajectories of other agents in beneficial ways rather than merely reacting to their current behaviors. Furthermore, the framework enables active influence by allowing agents to reason about how their actions affect not just the environment state but also the learning processes of other agents, facilitating sophisticated strategies like teaching, deception, or coordination that explicitly aim to influence other agents' policy updates. Under appropriate conditions, Active Markov Games provide theoretical guarantees about convergence to stationary periodic distributions, offering a more solid foundation for algorithm development than approaches without such guarantees. Finally, the Active Equilibrium concept generalizes traditional game-theoretic solution concepts, providing a more comprehensive framework for understanding stable multi-agent behaviors in learning settings.


\subsection{Practical Implementations}

Recent work has leveraged the Active Markov Game framework to develop practical algorithms for MARL. Notable examples include FURTHER (FUlly Reinforcing acTive influence witH averagE Reward) \cite{kim2022influencing}, which implements a policy gradient approach tailored to the average reward objective in Active Markov Games, combined with variational inference for estimating other agents' policy dynamics in a decentralized manner. Another significant approach is Meta-MAPG (Meta-Multiagent Policy Gradient) \cite{kim2021policy}, which integrates meta-learning with explicit modeling of other agents' learning processes, aligning with the influence-aware perspective of Active Markov Games. These algorithms have demonstrated superior performance compared to methods that neglect the learning dynamics of other agents, particularly in environments with high levels of non-stationarity.

\section{Extension to Partial Observability}

While Active Markov Games provide a powerful framework for addressing non-stationarity, they assume full observability of the environment state. In many real-world scenarios, agents have limited perception capabilities and cannot directly observe the complete state of the environment or the internal parameters of other agents. This partial observability introduces additional challenges for multi-agent learning. The Partially Observable Markov Decision Process (POMDP) \cite{kaelbling1998planning} extends MDPs to settings with partial observability by introducing observation functions. In the multi-agent context, this leads to Partially Observable Stochastic Games (POSGs) \cite{hansen2004dynamic} or Decentralized POMDPs (Dec-POMDPs) for cooperative settings \cite{bernstein2002complexity}. Extending Active Markov Games to partial observability settings represents a significant advancement in addressing the combined challenges of non-stationarity and limited information in multi-agent learning. The resulting framework, Partially Observable Active Markov Games, incorporates belief states and observation functions while maintaining the key benefits of Active Markov Games for modeling and influencing learning dynamics.

In partially observable settings, policy gradient methods require careful adaptation to handle the unavailability of the true state. Theoretical formulations, as established by Meuleau et al. \cite{meuleau1999pomdp}, extend the policy gradient theorem by conditioning policies on belief states rather than environmental states, resulting in $\nabla_{\theta^i} J(\theta^i) = \mathbb{E}_{s,b} \left[ \nabla_{\theta^i} \log \pi^i(a^i|b^i; \theta^i) Q^{\pi}(s,b^i,a^i) \right]$. However, this theoretical construction presents a practical challenge: agents don't have access to the true state distribution needed to compute this expectation. Practical implementations resolve this tension through multiple approaches: history-based methods condition policies on observation histories $h_t$ rather than belief states \cite{aberdeen2002pg}; recurrent network formulations implicitly encode history using recurrent architectures \cite{hausknecht2015drqn}; and belief-explicit methods maintain and update a belief distribution while estimating values only from observable quantities \cite{peng2022belief}. These approaches share a common principle of using trajectory sampling to estimate gradients without requiring knowledge of the true state or transition dynamics. In multi-agent settings with partial observability, the challenge compounds as agents must reason about others' belief states and learning dynamics, requiring sophisticated inference mechanisms to estimate other agents' policies and their evolution over time. For Partially Observable Active Markov Games, policy gradients can be formulated to account for the impact of current actions on both the future environmental states and the future policies of other agents, all while operating from belief states rather than full state observations.

\chapter{Proofs Regarding Average Returns}

This section provides the mathematical foundations and detailed proofs for the
key theoretical results in our Partially Observable Active Markov Game framework
with average return objectives. We establish the Markov properties of belief transitions
and joint processes, prove the existence and uniqueness of stochastically stable
distributions, and derive the policy gradient theorem for average return optimization.

\section{Markov Property of the Joint Process}
\label{appendix:stochasticallystable}We first establish the Markov properties of
belief state transitions and the joint process comprising states, belief states,
and policy parameters. These properties form the foundation for our convergence analysis.

\begin{lemma}[Markov Property of Belief Transitions]
    The sequence of belief states $\{b_{t}^{i}\}_{t \geq 0}$ for each agent $i$ forms
    a Markov process under a fixed policy $\pi$, meaning that the distribution of
    $b_{t+1}^{i}$ depends only on $b_{t}^{i}$, $a_{t}^{i}$, and $o_{t+1}^{i}$, and
    not on earlier beliefs, actions, or observations.
\end{lemma}

\begin{proof}
    By definition, the belief state $b_{t}^{i}$ at time $t$ incorporates all
    relevant information from the history of observations and actions up to time
    $t$. For Bayesian agents, given $b_{t}^{i}$, action $a_{t}^{i}$, and
    observation $o_{t+1}^{i}$, the belief update equation uniquely determines $b_{t+1}
            ^{i}$ through the rule:
    \begin{align}
        b^i_{t+1}(s') = \frac{O^i(o^i_{t+1}|s') \int_{s \in S} T(s'|s, \boldsymbol{a}_t) b^i_t(s) ds}{\int_{s'' \in S} O^i(o^i_{t+1}|s'') \int_{s \in S} T(s''|s, \boldsymbol{a}_t) b^i_t(s) ds ds''}
    \end{align}

    where $O_{i}(o_{t+1}^{i}|s')$ represents the probability of agent $i$ observing
    $o_{t+1}^{i}$ in state $s'$. This update depends only on $b_{t}^{i}$, $a_{t}^{i}$,
    and $o_{t+1}^{i}$, and not on the sequence of beliefs, actions, and observations
    that led to $b_{t}^{i}$. Therefore, the belief state transition satisfies the Markov
    property.
\end{proof}

\begin{remark}[Transformer-Based Belief Representation]
    In practice, transformer architectures can be used to represent and
    update belief states. In such cases, the belief state is updated
    through an attention-based mechanism of the form:
    \begin{equation}
        b_{t+1}^{i} = f_{{Transformer}}(b_{t}^{i}, o_{t+1}^{i}),
    \end{equation}

    that is parameterized by $\theta_{Transformer}^{i}$ and processes the latest
    observation and previous belief state through self-attention layers, satisfying the Markov property
    while capturing complex dependencies between observations and beliefs.
\end{remark}
Next, we establish the Markov property of the joint process.
\begin{lemma}[Markov Property of Joint Process]
    The joint process $(s_{t}, \boldsymbol{b}_{t}, \boldsymbol{\theta}_{t})$ forms
    a Markov process under periodic policy updates, meaning that the distribution
    of $(s_{t+1}, \boldsymbol{b}_{t+1}, \boldsymbol{\theta}_{t+1})$ depends only
    on $(s_{t}, \boldsymbol{b}_{t}, \boldsymbol{\theta}_{t})$ and not on earlier
    states.
\end{lemma}
\begin{proof}
    The state transitions are by definition only dependent on current state and current
    actions. The remaining parts can be analyzed as follows:

    \textbf{1. Belief
        state transition:} As established earlier, $\boldsymbol{b}_{t+1}$ depends only
    on $\boldsymbol{b}_{t}$, $\boldsymbol{a}_{t}$, and $\boldsymbol{o}_{t+1}$. Since
    actions $\boldsymbol{a}_{t}$ are drawn from policies $\pi(a_{t}^{i}|b_{t}^{i};
        \theta_{t}^{i})$ that depend only on $b_{t}^{i}$ and $\theta_{t}^{i}$ for each
    agent $i$, and observations $\boldsymbol{o}_{t+1}$ depend stochastically on
    the resulting state, the distribution of $\boldsymbol{b}_{t+1}$ depends only on
    $\boldsymbol{b}_{t}$ and $\boldsymbol{\theta}_{t}$.

    \textbf{2. Policy parameter
        update:} At every time step $t$, policy parameters are updated according to $U(
        \boldsymbol{\theta}_{t+1}|\boldsymbol{\theta}_{t}, \boldsymbol{\tau}_{t})$.
    Under the Markovian policy update assumption, the update depends only on $\theta
        _{t}$ and current trajectory $\tau_{t}$, which are functions of $s_{t}$, $\boldsymbol
        {b}_{t}$ and $\boldsymbol{\theta}_{t}$.Therefore, the joint transition
    probability
    $\mathbb{P}((s_{t+1}, \boldsymbol{b}_{t+1}, \boldsymbol{\theta}_{t+1})|(s_{t}, \boldsymbol
        {b}_{t}, \boldsymbol{\theta}_{t}))$
    depends only on the current state
    $(s_{t}, \boldsymbol{b}_{t}, \boldsymbol{\theta}_{t})$ and not on the history
    of states, establishing the Markov property.
\end{proof}
\section{Convergence}
Having established the Markov properties, we now analyze the convergence
behavior of the joint process to a unique stochastically stable distribution. This
convergence result is crucial for developing optimization objectives in our framework.
\begin{theorem}[Stochastically Stable Distribution]
    Under assumptions (1) and (2), as $\epsilon \to 0$, the perturbed joint processes
    defined by $\epsilon$-perturbed policy update functions converge to the unique
    stochastically stable distribution $\mu^{*}$ of the unperturbed joint process $(
        s_{t}, \boldsymbol{b}_{t}, \boldsymbol{\theta}_{t})$.
\end{theorem}

\begin{proof}
    We establish the convergence to the unique stochastically stable distribution by
    showing that a perturbed Markov process of a partially observable active Markov
    game is regular. First, let us introduce a small perturbation to the policy update
    functions:
    \begin{align}
        U^{\epsilon}_{i}(\theta^{i}_{t+1}|\theta^{i}_{t}, \tau^{i}_{t}) = (1 - \epsilon) U_{i}(\theta^{i}_{t+1}|\theta^{i}_{t}, \tau^{i}_{t}) + \epsilon \eta_{i}(\theta^{i}_{t+1})
    \end{align}

    where $\eta_{i}$ is a baseline distribution over $\Theta_{i}$ with full support,
    and $\epsilon > 0$ is a small constant. This perturbation ensures that from any
    policy parameter configuration $\boldsymbol{\theta}_{t}$, there is a positive
    probability of transitioning to any other configuration
    $\boldsymbol{\theta}_{t+1}$. The perturbed joint process $(s_{t}, \boldsymbol{b}
        _{t}, \boldsymbol{\theta}_{t})^{\epsilon}$ evolves according to the transition
    probability:
    \begin{align}
        \mathbb{P}(s_{t+1}, \boldsymbol{b}_{t+1}, \boldsymbol{\theta}_{t+1}|s_{t}, \boldsymbol{b}_{t}, \boldsymbol{\theta}_{t}) & = \sum_{\boldsymbol{a}_t \in A}\boldsymbol{\pi}(\boldsymbol{a}_{t}|\boldsymbol{b}_{t}; \boldsymbol{\theta}_{t}) \times T(s_{t+1}|s_{t}, \boldsymbol{a}_{t})                                     \\
                                                                                                                                & \times \sum_{\boldsymbol{o}_{t+1} \in O}\boldsymbol{O}(\boldsymbol{o}_{t+1}|s_{t+1}) \times \boldsymbol{U}^{\epsilon}(\boldsymbol{\theta}_{t+1}|\boldsymbol{\theta}_{t}, \boldsymbol{\tau}_{t}) \\
                                                                                                                                & \times \mathbb{I}(\boldsymbol{b}_{t+1}= \text{update}(\boldsymbol{b}_{t}, \boldsymbol{a}_{t}, \boldsymbol{o}_{t+1}))
    \end{align}

    where $\boldsymbol{\tau}_{t} = \{\boldsymbol{o}_{t}, \boldsymbol{a}_{t}, \boldsymbol
        {r}_{t}, \boldsymbol{o}_{t+1}\}$ is the joint trajectory,
    $\boldsymbol{r}_{t} = \{R^{i}(s_{t}, \boldsymbol{a}_{t})\}_{i \in I}$ is the
    vector of rewards, and $\text{update}(\boldsymbol{b}_{t}, \boldsymbol{a}_{t}, \boldsymbol
        {o}_{t+1})$ represents the belief update function for all agents. Suppose, for
    contradiction, that the perturbed Markov process is irregular, meaning it has multiple
    recurrent classes. Let $C_{1}$ and $C_{2}$ be two distinct recurrent classes
    of the joint process. We will show that there must exist a path with positive probability
    from any state in $C_{1}$ to any state in $C_{2}$, contradicting the
    assumption that they are distinct recurrent classes. Consider any $(s^{1}, \boldsymbol
        {b}^{1}, \boldsymbol{\theta}^{1}) \in C_{1}$ and $(s^{2}, \boldsymbol{b}^{2}, \boldsymbol
        {\theta}^{2}) \in C_{2}$. We construct a path from the first state to the
    second as follows:

    \textbf{1. Policy Parameter Transition:} Due to the
    $\epsilon$-perturbation in update functions, there is a positive probability of
    directly transitioning from $\boldsymbol{\theta}_{1}$ to any $\boldsymbol{\theta}
        ' \in \Theta$ in a single step, regardless of the trajectory. Specifically:
    \begin{align}
        \mathbb{P}(\boldsymbol{\theta}_{t+1}=\boldsymbol{\theta}'|\boldsymbol{\theta}_{t}=\boldsymbol{\theta}_{1}, \boldsymbol{\tau}_{t}) \geq \prod_{i \in I}\epsilon \cdot \eta_{i}(\theta'^{i}) > 0
    \end{align}

    Thus, there is a positive probability path from $\boldsymbol{\theta}_{1}$ to
    $\boldsymbol{\theta}_{2}$.

    \textbf{2. State Transition:} By the communicating
    state assumption, for any two states $s_{1}, s_{2} \in S$, there exists a
    sequence of joint actions $\boldsymbol{a}_{1}, \ldots, \boldsymbol{a}_{k}$
    such that:
    \begin{align}
        \mathbb{P}(s_{t+k}=s_{2}|s_{t}=s_{1}, \boldsymbol{a}_{t}=\boldsymbol{a}_{1}, \ldots, \boldsymbol{a}_{t+k-1}=\boldsymbol{a}_{k}) > 0
    \end{align}

    Given the relationship between policies and actions, this sequence of actions has
    positive probability under any policy parameters $\boldsymbol{\theta}$ and corresponding
    belief states $\boldsymbol{b}$ such that $\pi^{i}(a^{i}|b^{i}; \theta^{i}) > 0$
    for all required actions.

    \textbf{3. Belief State Transition:} By the communicating belief-state
    assumption, for any two belief states $\boldsymbol{b}_{1}, \boldsymbol{b}_{2} \in
        B$, there exists a sequence of observations
    $\boldsymbol{o}_{1}, \ldots, \boldsymbol{o}_{m}$ such that:
    \begin{align}
        \mathbb{P}(\boldsymbol{b}_{t+m}=\boldsymbol{b}_{2}|\boldsymbol{b}_{t}=\boldsymbol{b}_{1}, \boldsymbol{o}_{t+1}=\boldsymbol{o}_{1}, \ldots, \boldsymbol{o}_{t+m}=\boldsymbol{o}_{m}) > 0
    \end{align}

    Since observations depend on states, which can be influenced through actions,
    and actions are determined by policies, there is a positive probability path from
    $\boldsymbol{b}_{1}$ to $\boldsymbol{b}_{2}$ under appropriate state transitions
    and policy parameters. Combining these three components, we can construct a path
    with positive probability from $(s^{1}, \boldsymbol{b}^{1}, \boldsymbol{\theta}
        ^{1})$ to $(s^{2}, \boldsymbol{b}^{2}, \boldsymbol{\theta}^{2})$. This
    contradicts the assumption that $C_{1}$ and $C_{2}$ are distinct recurrent
    classes of the perturbed joint process. Since we have shown that the perturbed
    joint process has only one recurrent class, it is a regular Markov process. Following
    \citep{kim2022influencing,wicks2012algorithmcomputingstochasticallystable}, a regular Markov
    process on a finite state space possesses a unique stationary distribution
    $\mu^{\epsilon}$ to which it converges regardless of the initial state. Moreover,
    as $\epsilon \to 0$, the sequence of stationary distributions $\mu^{\epsilon}$
    converges to a limit $\mu^{*}$, which is the unique stochastically stable
    distribution of the original, unperturbed process. Therefore, under the given conditions,
    the joint process $(s_{t}, \boldsymbol{b}_{t}, \boldsymbol{\theta}_{t})$ in a
    partially observable active Markov game possesses a unique stochastically
    stable distribution $\mu^{*}$.
\end{proof}
\section{Policy Gradient Theorem}
Having established the convergence to a unique stochastically stable distribution,
we now derive the policy gradient theorem for optimization in our framework. This
theorem provides the mathematical foundation for gradient-based algorithms to maximize
the average return objective.
\label{appendix:avg_gradient}
\begin{theorem}[Partially Observable Active Average Reward Policy Gradient
        Theorem]
    The gradient of the active average reward objective
    with respect to agent $i$'s policy parameters $\theta^{i}$ in a partially observable
    setting is:
    \begin{align}
        \nabla_{\theta^i}J^{i}_{\pi}(\theta^{i}) & = \sum_{s, \boldsymbol{b}, \boldsymbol{\theta}}\mu_{\theta^i}(s, \boldsymbol{b}, \boldsymbol{\theta}) \sum_{a^i}\nabla_{\theta^i}\pi^{i}(a^{i}|b^{i}; \theta^{i}) \sum_{a^{\neg i}}\pi^{\neg i}(a^{\neg i}|b^{\neg i}; \theta^{\neg i}) q^{i}_{\theta^i}(s, \boldsymbol{b}, \boldsymbol{\theta}, \boldsymbol{a})
    \end{align}

    where the action-value function $q^{i}_{\theta^i}$ is defined as:
    \begin{align}
        q^{i}_{\theta^i}(s, \boldsymbol{b}, \boldsymbol{\theta}, \boldsymbol{a}) = \sum_{s'}T(s'|s, \boldsymbol{a}) \sum_{\boldsymbol{o}'}\boldsymbol{O}(\boldsymbol{o}'|s') \sum_{\boldsymbol{\theta}'}\boldsymbol{U}(\boldsymbol{\theta}'|\boldsymbol{\theta}, \boldsymbol{\tau}) \left[ R^{i}(s, \boldsymbol{a}) - \rho^{i}_{\theta^i}+ v^{i}_{\theta^i}(s', \text{update}(\boldsymbol{b}, \boldsymbol{a}, \boldsymbol{o}'), \boldsymbol{\theta}') \right]
    \end{align}

    with $\text{update}(\boldsymbol{b}, \boldsymbol{a}, \boldsymbol{o}')$ representing
    the belief update function.
\end{theorem}

\begin{proof}
    We first define the average reward objective in the partially observable setting:
    \begin{align}
        \max_{\theta^i}\rho^{i}_{\theta^i}(\boldsymbol{b}, \boldsymbol{\theta}) & := \max_{\theta^i}\lim_{T \to \infty}\mathbb{E}\left[ \frac{1}{T}\sum_{t=0}^{T} R^{i}(s_{t}, \boldsymbol{a}_{t}) \bigg| \begin{array}{c}\boldsymbol{b}_0= \boldsymbol{b}, \boldsymbol{\theta}_0= \boldsymbol{\theta}, \\ \boldsymbol{a}_{0:T} \sim \boldsymbol{\pi}(\cdot|\boldsymbol{b}_{0:T}; \boldsymbol{\theta}_{0:T}), s_{t+1} \sim T(\cdot|s_t, \boldsymbol{a}_t), \\ \boldsymbol{o}_{t+1} \sim \boldsymbol{O}(\cdot|s_{t+1}), \boldsymbol{\theta}_{t+1} \sim \boldsymbol{U}(\cdot|\boldsymbol{\theta}_t, \boldsymbol{\tau}_t)\end{array} \right] \\
                                                                                & = \max_{\theta^i}\sum_{s, \boldsymbol{b}, \boldsymbol{\theta}}b^{i}(s) \mu_{\theta^i}(s, \boldsymbol{b}, \boldsymbol{\theta}) \sum_{\boldsymbol{a}}\boldsymbol{\pi}(\boldsymbol{a}|\boldsymbol{b}; \boldsymbol{\theta}) R^{i}(s, \boldsymbol{a})
    \end{align}

    where $\mu_{\theta^i}(s, \boldsymbol{b}, \boldsymbol{\theta})$ is the stationary
    distribution of the joint process $(s_{t}, \boldsymbol{b}_{t}, \boldsymbol{\theta}
        _{t})$ under agent $i$'s policy parameterized by $\theta^{i}$. The
    differential value function $v^{i}_{\theta^i}$ represents the expected total
    difference between the accumulated rewards from state $s$, belief states $\boldsymbol
        {b}$, and policy parameters $\boldsymbol{\theta}$, and the average reward $\rho
        ^{i}_{\theta^i}$:
    \begin{align}
        v^{i}_{\theta^i}(s, \boldsymbol{b}, \boldsymbol{\theta}) = \lim_{T \to \infty}\mathbb{E}\left[ \sum_{t=0}^{T} \left(R^{i}(s_{t}, \boldsymbol{a}_{t}) - \rho^{i}_{\theta^i}\right) \bigg| \begin{array}{c}s_0=s, \boldsymbol{b}_0= \boldsymbol{b}, \boldsymbol{\theta}_0= \boldsymbol{\theta}, \\ \boldsymbol{a}_{0:T} \sim \boldsymbol{\pi}(\cdot|\boldsymbol{b}_{0:T}; \boldsymbol{\theta}_{0:T}), s_{t+1} \sim T(\cdot|s_t, \boldsymbol{a}_t), \\ \boldsymbol{o}_{t+1} \sim \boldsymbol{O}(\cdot|s_{t+1}), \boldsymbol{\theta}_{t+1} \sim \boldsymbol{U}(\cdot|\boldsymbol{\theta}_t, \boldsymbol{\tau}_t)\end{array} \right]
    \end{align}

    Following the Bellman equation derivation principles, we can express
    $v^{i}_{\theta^i}(s, \boldsymbol{b}, \boldsymbol{\theta})$ recursively:
    \begin{align}
        v^{i}_{\theta^i}(s, \boldsymbol{b}, \boldsymbol{\theta}) & = \mathbb{E}_{\boldsymbol{a} \sim \boldsymbol{\pi}(\cdot|\boldsymbol{b};\boldsymbol{\theta}), s' \sim T(\cdot|s,\boldsymbol{a}), \boldsymbol{o}' \sim \boldsymbol{O}(\cdot|s'), \boldsymbol{\theta}' \sim \boldsymbol{U}(\cdot|\boldsymbol{\theta}, \boldsymbol{\tau})}\left[ R^{i}(s, \boldsymbol{a}) - \rho^{i}_{\theta^i}+ v^{i}_{\theta^i}(s', \text{update}(\boldsymbol{b}, \boldsymbol{a}, \boldsymbol{o}'), \boldsymbol{\theta}') \right] \\
                                                                 & = \sum_{a^i}\pi^{i}(a^{i}|b^{i}; \theta^{i}) \sum_{a^{\neg i}}\pi^{\neg i}(a^{\neg i}|b^{\neg i}; \theta^{\neg i}) \sum_{s'}T(s'|s, \boldsymbol{a}) \sum_{\boldsymbol{o}'}\boldsymbol{O}(\boldsymbol{o}'|s') \sum_{\boldsymbol{\theta}'}\boldsymbol{U}(\boldsymbol{\theta}'|\boldsymbol{\theta}, \boldsymbol{\tau})                                                                                                                              \\
                                                                 & \quad \left[ R^{i}(s, \boldsymbol{a}) - \rho^{i}_{\theta^i}+ v^{i}_{\theta^i}(s', \text{update}(\boldsymbol{b}, \boldsymbol{a}, \boldsymbol{o}'), \boldsymbol{\theta}') \right]
    \end{align}

    where $\text{update}(\boldsymbol{b}, \boldsymbol{a}, \boldsymbol{o}')$
    represents the belief update function that updates the belief state based on
    the current belief, action, and new observation. We now define the action-value
    function $q^{i}_{\theta^i}$ for the partially observable setting:
    \begin{align}
        q^{i}_{\theta^i}(s, \boldsymbol{b}, \boldsymbol{\theta}, \boldsymbol{a}) = \sum_{s'}T(s'|s, \boldsymbol{a}) \sum_{\boldsymbol{o}'}\boldsymbol{O}(\boldsymbol{o}'|s') \sum_{\boldsymbol{\theta}'}\boldsymbol{U}(\boldsymbol{\theta}'|\boldsymbol{\theta}, \boldsymbol{\tau}) \left[ R^{i}(s, \boldsymbol{a}) - \rho^{i}_{\theta^i}+ v^{i}_{\theta^i}(s', \text{update}(\boldsymbol{b}, \boldsymbol{a}, \boldsymbol{o}'), \boldsymbol{\theta}') \right]
    \end{align}

    Using this definition, we can rewrite the differential value function as:
    \begin{align}
        v^{i}_{\theta^i}(s, \boldsymbol{b}, \boldsymbol{\theta}) = \sum_{a^i}\pi^{i}(a^{i}|b^{i}; \theta^{i}) \sum_{a^{\neg i}}\pi^{\neg i}(a^{\neg i}|b^{\neg i}; \theta^{\neg i}) q^{i}_{\theta^i}(s, \boldsymbol{b}, \boldsymbol{\theta}, \boldsymbol{a})
    \end{align}

    To derive the policy gradient, we begin by computing the gradient of the differential
    value function with respect to $\theta^{i}$:
    \begin{align}
        \nabla_{\theta^i}v^{i}_{\theta^i}(s, \boldsymbol{b}, \boldsymbol{\theta}) & = \nabla_{\theta^i}\left[ \sum_{a^i}\pi^{i}(a^{i}|b^{i}; \theta^{i}) \sum_{a^{\neg i}}\pi^{\neg i}(a^{\neg i}|b^{\neg i}; \theta^{\neg i}) q^{i}_{\theta^i}(s, \boldsymbol{b}, \boldsymbol{\theta}, \boldsymbol{a}) \right] \\
                                                                                  & = \sum_{a^i}\nabla_{\theta^i}\pi^{i}(a^{i}|b^{i}; \theta^{i}) \sum_{a^{\neg i}}\pi^{\neg i}(a^{\neg i}|b^{\neg i}; \theta^{\neg i}) q^{i}_{\theta^i}(s, \boldsymbol{b}, \boldsymbol{\theta}, \boldsymbol{a}) +              \\
                                                                                  & \quad \sum_{a^i}\pi^{i}(a^{i}|b^{i}; \theta^{i}) \sum_{a^{\neg i}}\pi^{\neg i}(a^{\neg i}|b^{\neg i}; \theta^{\neg i}) \nabla_{\theta^i}q^{i}_{\theta^i}(s, \boldsymbol{b}, \boldsymbol{\theta}, \boldsymbol{a})
    \end{align}

    We expand the gradient of the action-value function:
    \begin{align}
        \nabla_{\theta^i}q^{i}_{\theta^i}(s, \boldsymbol{b}, \boldsymbol{\theta}, \boldsymbol{a}) & = \nabla_{\theta^i}\left[ \sum_{s'}T(s'|s, \boldsymbol{a}) \sum_{\boldsymbol{o}'}\boldsymbol{O}(\boldsymbol{o}'|s') \right.                                                                                                                                                                                                                                           \\
                                                                                                  & \left. \sum_{\boldsymbol{\theta}'}\boldsymbol{U}(\boldsymbol{\theta}'|\boldsymbol{\theta}, \boldsymbol{\tau}) \left[ R^{i}(s, \boldsymbol{a}) - \rho^{i}_{\theta^i}+ v^{i}_{\theta^i}(s', \text{update}(\boldsymbol{b}, \boldsymbol{a}, \boldsymbol{o}'), \boldsymbol{\theta}') \right] \right]                                                                       \\
                                                                                                  & = -\nabla_{\theta^i}\rho^{i}_{\theta^i}+ \sum_{s'}T(s'|s, \boldsymbol{a}) \sum_{\boldsymbol{o}'}\boldsymbol{O}(\boldsymbol{o}'|s') \sum_{\boldsymbol{\theta}'}\boldsymbol{U}(\boldsymbol{\theta}'|\boldsymbol{\theta}, \boldsymbol{\tau}) \nabla_{\theta^i}v^{i}_{\theta^i}(s', \text{update}(\boldsymbol{b}, \boldsymbol{a}, \boldsymbol{o}'), \boldsymbol{\theta}')
    \end{align}

    Substituting back into the original expression, we get:
    \begin{align}
        \nabla_{\theta^i}v^{i}_{\theta^i}(s, \boldsymbol{b}, \boldsymbol{\theta}) & = \sum_{a^i}\nabla_{\theta^i}\pi^{i}(a^{i}|b^{i}; \theta^{i}) \sum_{a^{\neg i}}\pi^{\neg i}(a^{\neg i}|b^{\neg i}; \theta^{\neg i}) q^{i}_{\theta^i}(s, \boldsymbol{b}, \boldsymbol{\theta}, \boldsymbol{a})                             \\
                                                                                  & \quad - \sum_{a^i}\pi^{i}(a^{i}|b^{i}; \theta^{i}) \sum_{a^{\neg i}}\pi^{\neg i}(a^{\neg i}|b^{\neg i}; \theta^{\neg i}) \nabla_{\theta^i}\rho^{i}_{\theta^i}                                                                            \\
                                                                                  & \quad + \sum_{a^i}\pi^{i}(a^{i}|b^{i}; \theta^{i}) \sum_{a^{\neg i}}\pi^{\neg i}(a^{\neg i}|b^{\neg i}; \theta^{\neg i}) \sum_{s'}T(s'|s, \boldsymbol{a}) \sum_{\boldsymbol{o}'}\boldsymbol{O}(\boldsymbol{o}'|s')                       \\
                                                                                  & \quad \sum_{\boldsymbol{\theta}'}\boldsymbol{U}(\boldsymbol{\theta}'|\boldsymbol{\theta}, \boldsymbol{\tau}) \nabla_{\theta^i}v^{i}_{\theta^i}(s', \text{update}(\boldsymbol{b}, \boldsymbol{a}, \boldsymbol{o}'), \boldsymbol{\theta}')
    \end{align}

    Rearranging terms:
    \begin{align}
        \nabla_{\theta^i}\rho^{i}_{\theta^i} & = \sum_{a^i}\nabla_{\theta^i}\pi^{i}(a^{i}|b^{i}; \theta^{i}) \sum_{a^{\neg i}}\pi^{\neg i}(a^{\neg i}|b^{\neg i}; \theta^{\neg i}) q^{i}_{\theta^i}(s, \boldsymbol{b}, \boldsymbol{\theta}, \boldsymbol{a})                             \\
                                             & \quad + \sum_{a^i}\pi^{i}(a^{i}|b^{i}; \theta^{i}) \sum_{a^{\neg i}}\pi^{\neg i}(a^{\neg i}|b^{\neg i}; \theta^{\neg i}) \sum_{s'}T(s'|s, \boldsymbol{a}) \sum_{\boldsymbol{o}'}\boldsymbol{O}(\boldsymbol{o}'|s')                       \\
                                             & \quad \sum_{\boldsymbol{\theta}'}\boldsymbol{U}(\boldsymbol{\theta}'|\boldsymbol{\theta}, \boldsymbol{\tau}) \nabla_{\theta^i}v^{i}_{\theta^i}(s', \text{update}(\boldsymbol{b}, \boldsymbol{a}, \boldsymbol{o}'), \boldsymbol{\theta}') \\
                                             & \quad - \nabla_{\theta^i}v^{i}_{\theta^i}(s, \boldsymbol{b}, \boldsymbol{\theta})
    \end{align}

    Now, we define the transition operator that maps the expectation of a function
    from one time step to the next:
    \begin{align}
        \Psi f(s, \boldsymbol{b}, \boldsymbol{\theta}) & = \sum_{a^i}\pi^{i}(a^{i}|b^{i}; \theta^{i}) \sum_{a^{\neg i}}\pi^{\neg i}(a^{\neg i}|b^{\neg i}; \theta^{\neg i})                                                                                                                                                                                 \\
                                                       & \quad \sum_{s'}T(s'|s, \boldsymbol{a}) \sum_{\boldsymbol{o}'}\boldsymbol{O}(\boldsymbol{o}'|s') \sum_{\boldsymbol{\theta}'}\boldsymbol{U}(\boldsymbol{\theta}'|\boldsymbol{\theta}, \boldsymbol{\tau}) f(s', \text{update}(\boldsymbol{b}, \boldsymbol{a}, \boldsymbol{o}'), \boldsymbol{\theta}')
    \end{align}

    We can rewrite our expression as:
    \begin{align}
        \nabla_{\theta^i}\rho^{i}_{\theta^i} & = \sum_{a^i}\nabla_{\theta^i}\pi^{i}(a^{i}|b^{i}; \theta^{i}) \sum_{a^{\neg i}}\pi^{\neg i}(a^{\neg i}|b^{\neg i}; \theta^{\neg i}) q^{i}_{\theta^i}(s, \boldsymbol{b}, \boldsymbol{\theta}, \boldsymbol{a}) + \Psi(\nabla_{\theta^i}v^{i}_{\theta^i})(s, \boldsymbol{b}, \boldsymbol{\theta}) - \nabla_{\theta^i}v^{i}_{\theta^i}(s, \boldsymbol{b}, \boldsymbol{\theta})
    \end{align}

    Now, we take the expectation with respect to the stationary distribution
    $\mu_{\theta^i}(s, \boldsymbol{b}, \boldsymbol{\theta})$:
    \begin{align}
        \sum_{s, \boldsymbol{b}, \boldsymbol{\theta}}\mu_{\theta^i}(s, \boldsymbol{b}, \boldsymbol{\theta}) \nabla_{\theta^i}\rho^{i}_{\theta^i} & = \sum_{s, \boldsymbol{b}, \boldsymbol{\theta}}\mu_{\theta^i}(s, \boldsymbol{b}, \boldsymbol{\theta}) \sum_{a^i}\nabla_{\theta^i}\pi^{i}(a^{i}|b^{i}; \theta^{i}) \sum_{a^{\neg i}}\pi^{\neg i}(a^{\neg i}|b^{\neg i}; \theta^{\neg i}) q^{i}_{\theta^i}(s, \boldsymbol{b}, \boldsymbol{\theta}, \boldsymbol{a})                                                            \\
                                                                                                                                                 & \quad + \sum_{s, \boldsymbol{b}, \boldsymbol{\theta}}\mu_{\theta^i}(s, \boldsymbol{b}, \boldsymbol{\theta}) \Psi(\nabla_{\theta^i}v^{i}_{\theta^i})(s, \boldsymbol{b}, \boldsymbol{\theta}) - \sum_{s, \boldsymbol{b}, \boldsymbol{\theta}}\mu_{\theta^i}(s, \boldsymbol{b}, \boldsymbol{\theta}) \nabla_{\theta^i}v^{i}_{\theta^i}(s, \boldsymbol{b}, \boldsymbol{\theta})
    \end{align}

    By the definition of the stationary distribution, we have:
    \begin{align}
        \sum_{s, \boldsymbol{b}, \boldsymbol{\theta}}\mu_{\theta^i}(s, \boldsymbol{b}, \boldsymbol{\theta}) \Psi f(s, \boldsymbol{b}, \boldsymbol{\theta}) = \sum_{s, \boldsymbol{b}, \boldsymbol{\theta}}\mu_{\theta^i}(s, \boldsymbol{b}, \boldsymbol{\theta}) f(s, \boldsymbol{b}, \boldsymbol{\theta})
    \end{align}

    Therefore, the second and third terms in our expression cancel out, giving the
    final policy gradient theorem:
    \begin{align}
        \nabla_{\theta^i}J^{i}_{\pi}(\theta^{i}) := \nabla_{\theta^i}\rho^{i}_{\theta^i} & = \sum_{s, \boldsymbol{b}, \boldsymbol{\theta}}\mu_{\theta^i}(s, \boldsymbol{b}, \boldsymbol{\theta}) \sum_{a^i}\nabla_{\theta^i}\pi^{i}(a^{i}|b^{i}; \theta^{i}) \sum_{a^{\neg i}}\pi^{\neg i}(a^{\neg i}|b^{\neg i}; \theta^{\neg i}) q^{i}_{\theta^i}(s, \boldsymbol{b}, \boldsymbol{\theta}, \boldsymbol{a})
    \end{align}
\end{proof}

\chapter{Proofs Regarding Discounted Returns}
\label{appendix:discounted_proofs} We establish the Partially Observable Active Discounted
Return Policy Gradient Theorem through a sequence of lemmas that build the necessary
mathematical foundation.
\section{Transition Operator and Its Adjoint}
To establish a rigorous framework for analyzing discounted returns in Partially
Observable Active Markov Games, we need to formalize how value functions evolve over
time and how probability distributions propagate through the system. This dual
perspective is captured by two fundamental operators: the transition operator and
its adjoint.

\subsection{Definitions and Duality}

We begin by defining the inner product between functions and measures, which forms
the foundation for the duality relationship between our operators.

\begin{definition}[Inner Product between Functions and Measures]
    For a bounded measurable function $f \in \mathcal{B}(S \times \boldsymbol{B}\times
        \boldsymbol{\Theta})$ and a finite measure $\mu \in \mathcal{M}(S \times \boldsymbol
        {B}\times \boldsymbol{\Theta})$, their inner product is defined as the
    Lebesgue integral:
    \begin{equation}
        \langle f, \mu \rangle = \int_{S \times \boldsymbol{B} \times \boldsymbol{\Theta}}
        f(s, \boldsymbol{b}, \boldsymbol{\theta}) \, d\mu(s, \boldsymbol{b}, \boldsymbol
        {\theta})
    \end{equation}
\end{definition}

This inner product captures the expected value of function $f$ with respect to
measure $\mu$, allowing us to express value functions as expectations with respect
to appropriate measures.

\begin{definition}[Transition Operator and Its Adjoint]
    \hfill
    \begin{enumerate}
        \item The \textbf{transition operator}
              $\Psi: \mathcal{B}(S \times \boldsymbol{B}\times \boldsymbol{\Theta}) \to \mathcal{B}
                  (S \times \boldsymbol{B}\times \boldsymbol{\Theta})$
              maps a bounded measurable function to its expected value after one
              transition:
              \begin{align}
                  (\Psi f)(s, \boldsymbol{b}, \boldsymbol{\theta}) & = \sum_{a^i}\pi^{i}(a^{i}|b^{i}; \theta^{i}) \sum_{a^{\neg i}}\pi^{\neg i}(a^{\neg i}|b^{\neg i}; \theta^{\neg i}) \sum_{s'}T(s'|s, \boldsymbol{a})                                                                                                               \\
                                                                   & \quad \sum_{\boldsymbol{o}'}\boldsymbol{O}(\boldsymbol{o}'|s') \sum_{\boldsymbol{\theta}'}\boldsymbol{U}(\boldsymbol{\theta}'|\boldsymbol{\theta}, \boldsymbol{\tau}) f(s', \text{update}(\boldsymbol{b}, \boldsymbol{a}, \boldsymbol{o}'), \boldsymbol{\theta}')
              \end{align}

              where $\text{update}(\boldsymbol{b}, \boldsymbol{a}, \boldsymbol{o}')$
              represents the belief update function.

        \item The \textbf{adjoint operator}
              $\Psi^{*}: \mathcal{M}(S \times \boldsymbol{B}\times \boldsymbol{\Theta}) \rightarrow
                  \mathcal{M}(S \times \boldsymbol{B}\times \boldsymbol{\Theta})$
              is defined on the space of finite measures such that:
              \begin{equation}
                  \langle \Psi f, \mu \rangle = \langle f, \Psi^{*} \mu \rangle
              \end{equation}

              for all
              $f \in \mathcal{B}(S \times \boldsymbol{B}\times \boldsymbol{\Theta})$ and
              $\mu \in \mathcal{M}(S \times \boldsymbol{B}\times \boldsymbol{\Theta})$.
    \end{enumerate}
\end{definition}

Intuitively, $\Psi$ describes how function expectations evolve forward in time,
meanwhile, $\Psi^{*}$ describes how probability distributions evolve in time,
capturing the propagation of visitation probabilities through the system. The duality
relationship ensures that expected values can be computed either by applying $\Psi$
to functions or $\Psi^{*}$ to measures.

\subsection{Properties of the Operators}

These operators possess several important properties that facilitate our
analysis of discounted returns:

\begin{lemma}[Properties of the Transition Operator and Its Adjoint]
    The transition operator $\Psi$ and its adjoint $\Psi^{*}$ satisfy the following
    properties:

    \begin{enumerate}
        \item \textbf{Linearity:}
              \begin{itemize}
                  \item For the operator: For any functions
                        $f, g \in \mathcal{B}(S \times \boldsymbol{B}\times \boldsymbol{\Theta}
                            )$
                        and constants $\alpha, \beta \in \mathbb{R}$:
                        \begin{align}
                            \Psi(\alpha f + \beta g) = \alpha \Psi f + \beta \Psi g
                        \end{align}

                  \item For the adjoint: For any measures
                        $\mu, \nu \in \mathcal{M}(S \times \boldsymbol{B}\times \boldsymbol{\Theta}
                            )$
                        and constants $\alpha, \beta \in \mathbb{R}$:
                        \begin{align}
                            \Psi^{*}(\alpha\mu + \beta\nu) = \alpha\Psi^{*}\mu + \beta\Psi^{*}\nu
                        \end{align}
              \end{itemize}

        \item \textbf{Positivity Preservation:}
              \begin{itemize}
                  \item For the operator: If $f \geq 0$, then $\Psi f \geq 0$.

                  \item For the adjoint: If $\mu$ is a non-negative measure, then $\Psi^{*}
                            \mu$ is also non-negative.
              \end{itemize}

        \item \textbf{Preservation of Total Measure:}
              \begin{itemize}
                  \item For the operator: The operator preserves the constant function
                        $\mathbf{1}$:
                        \begin{align}
                            \Psi \mathbf{1}= \mathbf{1}
                        \end{align}

                        where $\mathbf{1}$ represents the function that equals 1 everywhere.

                  \item For the adjoint: For any measure
                        $\mu \in \mathcal{M}(S \times \boldsymbol{B}\times \boldsymbol{\Theta})$:
                        \begin{align}
                            (\Psi^{*}\mu)(S \times \boldsymbol{B}\times \boldsymbol{\Theta}) = \mu(S \times \boldsymbol{B}\times \boldsymbol{\Theta})
                        \end{align}
              \end{itemize}

        \item \textbf{Bound Preservation:} For any bounded function $f$, $\Psi f$ is
              bounded with:
              \begin{align}
                  \|\Psi f\|_{\infty}\leq \|f\|_{\infty}
              \end{align}

              where $\|f\|_{\infty}= \sup_{s,\boldsymbol{b},\boldsymbol{\theta}}|f(s,\boldsymbol
                  {b},\boldsymbol{\theta})|$ is the supremum norm.

        \item \textbf{Explicit Form of the Adjoint:} For any measurable set
              $A \subset S \times \boldsymbol{B}\times \boldsymbol{\Theta}$:
              \begin{align}
                  (\Psi^{*}\mu)(A) & = \int_{S \times \boldsymbol{B} \times \boldsymbol{\Theta}}\sum_{a^i}\pi^{i}(a^{i}|b^{i}; \theta^{i}) \sum_{a^{\neg i}}\pi^{\neg i}(a^{\neg i}|b^{\neg i}; \theta^{\neg i})                                                                                                                                                                                     \\
                                   & \quad \sum_{s'}T(s'|s, \boldsymbol{a}) \sum_{\boldsymbol{o}'}\boldsymbol{O}(\boldsymbol{o}'|s') \sum_{\boldsymbol{\theta}'}\boldsymbol{U}(\boldsymbol{\theta}'|\boldsymbol{\theta}, \boldsymbol{\tau}) \mathbf{1}_{A}(s', \text{update}(\boldsymbol{b}, \boldsymbol{a}, \boldsymbol{o}'), \boldsymbol{\theta}') \, d\mu(s, \boldsymbol{b}, \boldsymbol{\theta})
              \end{align}

              where $\mathbf{1}_{A}$ is the indicator function for set $A$.
    \end{enumerate}
\end{lemma}

\begin{proof}
    We prove each property in turn:

    \textbf{1. Linearity:}

    For the operator: For any $(s, \boldsymbol{b}, \boldsymbol{\theta})$, we have:
    \begin{align}
         & \Psi(\alpha f + \beta g)(s, \boldsymbol{b}, \boldsymbol{\theta})                                                                                                                                                            \\
         & = \sum_{a^i}\pi^{i}(a^{i}|b^{i}; \theta^{i}) \sum_{a^{\neg i}}\pi^{\neg i}(a^{\neg i}|b^{\neg i}; \theta^{\neg i}) \sum_{s'}T(s'|s, \boldsymbol{a}) \sum_{\boldsymbol{o}'}\boldsymbol{O}(\boldsymbol{o}'|s')                \\
         & \quad \sum_{\boldsymbol{\theta}'}\boldsymbol{U}(\boldsymbol{\theta}'|\boldsymbol{\theta}, \boldsymbol{\tau}) [\alpha f + \beta g](s', \text{update}(\boldsymbol{b}, \boldsymbol{a}, \boldsymbol{o}'), \boldsymbol{\theta}') \\
         & = \alpha \sum_{a^i}\pi^{i}(a^{i}|b^{i}; \theta^{i}) \sum_{a^{\neg i}}\pi^{\neg i}(a^{\neg i}|b^{\neg i}; \theta^{\neg i}) \sum_{s'}T(s'|s, \boldsymbol{a}) \sum_{\boldsymbol{o}'}\boldsymbol{O}(\boldsymbol{o}'|s')         \\
         & \quad \sum_{\boldsymbol{\theta}'}\boldsymbol{U}(\boldsymbol{\theta}'|\boldsymbol{\theta}, \boldsymbol{\tau}) f(s', \text{update}(\boldsymbol{b}, \boldsymbol{a}, \boldsymbol{o}'), \boldsymbol{\theta}')                    \\
         & \quad + \beta \sum_{a^i}\pi^{i}(a^{i}|b^{i}; \theta^{i}) \sum_{a^{\neg i}}\pi^{\neg i}(a^{\neg i}|b^{\neg i}; \theta^{\neg i}) \sum_{s'}T(s'|s, \boldsymbol{a}) \sum_{\boldsymbol{o}'}\boldsymbol{O}(\boldsymbol{o}'|s')    \\
         & \quad \sum_{\boldsymbol{\theta}'}\boldsymbol{U}(\boldsymbol{\theta}'|\boldsymbol{\theta}, \boldsymbol{\tau}) g(s', \text{update}(\boldsymbol{b}, \boldsymbol{a}, \boldsymbol{o}'), \boldsymbol{\theta}')                    \\
         & = \alpha (\Psi f)(s, \boldsymbol{b}, \boldsymbol{\theta}) + \beta (\Psi g)(s, \boldsymbol{b}, \boldsymbol{\theta})
    \end{align}

    For the adjoint: For any measurable function
    $f \in \mathcal{B}(S \times \boldsymbol{B}\times \boldsymbol{\Theta})$:
    \begin{align}
        \langle f, \Psi^{*}(\alpha\mu + \beta\nu) \rangle & = \langle \Psi f, \alpha\mu + \beta\nu \rangle                               \\
                                                          & = \alpha\langle \Psi f, \mu \rangle + \beta\langle \Psi f, \nu \rangle       \\
                                                          & = \alpha\langle f, \Psi^{*}\mu \rangle + \beta\langle f, \Psi^{*}\nu \rangle \\
                                                          & = \langle f, \alpha\Psi^{*}\mu + \beta\Psi^{*}\nu \rangle
    \end{align}

    Since this holds for any measurable function $f$, we have $\Psi^{*}(\alpha\mu +
        \beta\nu) = \alpha\Psi^{*}\mu + \beta\Psi^{*}\nu$ by the uniqueness of the representing
    measure.

    \textbf{2. Positivity Preservation:}

    For the operator: If $f \geq 0$, then for any $(s, \boldsymbol{b}, \boldsymbol{\theta}
        )$:
    \begin{align}
        (\Psi f)(s, \boldsymbol{b}, \boldsymbol{\theta}) & = \sum_{a^i}\pi^{i}(a^{i}|b^{i}; \theta^{i}) \sum_{a^{\neg i}}\pi^{\neg i}(a^{\neg i}|b^{\neg i}; \theta^{\neg i}) \sum_{s'}T(s'|s, \boldsymbol{a}) \sum_{\boldsymbol{o}'}\boldsymbol{O}(\boldsymbol{o}'|s') \\
                                                         & \quad \sum_{\boldsymbol{\theta}'}\boldsymbol{U}(\boldsymbol{\theta}'|\boldsymbol{\theta}, \boldsymbol{\tau}) f(s', \text{update}(\boldsymbol{b}, \boldsymbol{a}, \boldsymbol{o}'), \boldsymbol{\theta}')
    \end{align}

    Since $f \geq 0$ and all probability distributions $\pi^{i}$, $\pi^{\neg i}$, $T$,
    $\boldsymbol{O}$, and $\boldsymbol{U}$ are non-negative, the entire sum
    consists of non-negative terms, making $(\Psi f)(s, \boldsymbol{b}, \boldsymbol
        {\theta}) \geq 0$.

    For the adjoint: If $\mu$ is non-negative, then for any non-negative measurable
    function $f \geq 0$:
    \begin{align}
        \langle f, \Psi^{*}\mu \rangle = \langle \Psi f, \mu \rangle \geq 0
    \end{align}

    where the inequality follows from the positivity preservation of $\Psi$ and the
    fact that $\mu$ is non-negative. Since this holds for all non-negative
    functions $f$, $\Psi^{*}\mu$ must be a non-negative measure.

    \textbf{3. Preservation of Total Measure:}

    For the operator: Let $\mathbf{1}$ be the constant function that equals 1
    everywhere. Then:
    \begin{align}
        (\Psi \mathbf{1})(s, \boldsymbol{b}, \boldsymbol{\theta}) & = \sum_{a^i}\pi^{i}(a^{i}|b^{i}; \theta^{i}) \sum_{a^{\neg i}}\pi^{\neg i}(a^{\neg i}|b^{\neg i}; \theta^{\neg i}) \sum_{s'}T(s'|s, \boldsymbol{a}) \sum_{\boldsymbol{o}'}\boldsymbol{O}(\boldsymbol{o}'|s') \\
                                                                  & \quad \sum_{\boldsymbol{\theta}'}\boldsymbol{U}(\boldsymbol{\theta}'|\boldsymbol{\theta}, \boldsymbol{\tau}) \cdot 1
    \end{align}

    Since each probability distribution sums to 1, applying them sequentially
    yields:
    \begin{align}
        (\Psi \mathbf{1})(s, \boldsymbol{b}, \boldsymbol{\theta}) & = \sum_{a^i}\pi^{i}(a^{i}|b^{i}; \theta^{i}) \sum_{a^{\neg i}}\pi^{\neg i}(a^{\neg i}|b^{\neg i}; \theta^{\neg i}) \sum_{s'}T(s'|s, \boldsymbol{a}) \sum_{\boldsymbol{o}'}\boldsymbol{O}(\boldsymbol{o}'|s') \cdot 1 \\
                                                                  & = \sum_{a^i}\pi^{i}(a^{i}|b^{i}; \theta^{i}) \sum_{a^{\neg i}}\pi^{\neg i}(a^{\neg i}|b^{\neg i}; \theta^{\neg i}) \sum_{s'}T(s'|s, \boldsymbol{a}) \cdot 1                                                          \\
                                                                  & = \sum_{a^i}\pi^{i}(a^{i}|b^{i}; \theta^{i}) \sum_{a^{\neg i}}\pi^{\neg i}(a^{\neg i}|b^{\neg i}; \theta^{\neg i}) \cdot 1                                                                                           \\
                                                                  & = \sum_{a^i}\pi^{i}(a^{i}|b^{i}; \theta^{i}) \cdot 1 = 1
    \end{align}

    For the adjoint: Setting $f = \mathbf{1}$ in the adjoint relationship:
    \begin{align}
        \langle \mathbf{1}, \Psi^{*}\mu \rangle & = \langle \Psi \mathbf{1}, \mu \rangle = \langle \mathbf{1}, \mu \rangle
    \end{align}

    where we used the unit preservation property of $\Psi$. This gives:
    \begin{align}
        (\Psi^{*}\mu)(S \times \boldsymbol{B}\times \boldsymbol{\Theta}) = \mu(S \times \boldsymbol{B}\times \boldsymbol{\Theta})
    \end{align}

    \textbf{4. Bound Preservation:}

    For any bounded function $f$ with
    $\|f\|_{\infty}= \sup_{s,\boldsymbol{b},\boldsymbol{\theta}}|f(s,\boldsymbol{b}
        ,\boldsymbol{\theta})|$:
    \begin{align}
        |(\Psi f)(s, \boldsymbol{b}, \boldsymbol{\theta})| & = \left| \sum_{a^i}\pi^{i}(a^{i}|b^{i}; \theta^{i}) \sum_{a^{\neg i}}\pi^{\neg i}(a^{\neg i}|b^{\neg i}; \theta^{\neg i}) \sum_{s'}T(s'|s, \boldsymbol{a}) \sum_{\boldsymbol{o}'}\boldsymbol{O}(\boldsymbol{o}'|s') \right.   \\
                                                           & \quad \left. \sum_{\boldsymbol{\theta}'}\boldsymbol{U}(\boldsymbol{\theta}'|\boldsymbol{\theta}, \boldsymbol{\tau}) f(s', \text{update}(\boldsymbol{b}, \boldsymbol{a}, \boldsymbol{o}'), \boldsymbol{\theta}') \right|       \\
                                                           & \leq \sum_{a^i}\pi^{i}(a^{i}|b^{i}; \theta^{i}) \sum_{a^{\neg i}}\pi^{\neg i}(a^{\neg i}|b^{\neg i}; \theta^{\neg i}) \sum_{s'}T(s'|s, \boldsymbol{a}) \sum_{\boldsymbol{o}'}\boldsymbol{O}(\boldsymbol{o}'|s')               \\
                                                           & \quad \sum_{\boldsymbol{\theta}'}\boldsymbol{U}(\boldsymbol{\theta}'|\boldsymbol{\theta}, \boldsymbol{\tau}) \left|f(s', \text{update}(\boldsymbol{b}, \boldsymbol{a}, \boldsymbol{o}'), \boldsymbol{\theta}') \right|        \\
                                                           & \leq \|f\|_{\infty}\sum_{a^i}\pi^{i}(a^{i}|b^{i}; \theta^{i}) \sum_{a^{\neg i}}\pi^{\neg i}(a^{\neg i}|b^{\neg i}; \theta^{\neg i}) \sum_{s'}T(s'|s, \boldsymbol{a}) \sum_{\boldsymbol{o}'}\boldsymbol{O}(\boldsymbol{o}'|s') \\
                                                           & \quad \sum_{\boldsymbol{\theta}'}\boldsymbol{U}(\boldsymbol{\theta}'|\boldsymbol{\theta}, \boldsymbol{\tau}) = \|f\|_{\infty}\cdot 1 = \|f\|_{\infty}
    \end{align}

    Taking the supremum over all $(s, \boldsymbol{b}, \boldsymbol{\theta})$, we get
    $\|\Psi f\|_{\infty}\leq \|f\|_{\infty}$.

    \textbf{5. Explicit Form of the Adjoint:}

    For any measurable set $A \subset S \times \boldsymbol{B}\times \boldsymbol{\Theta}$:
    \begin{align}
        (\Psi^{*}\mu)(A) & = \langle \mathbf{1}_{A}, \Psi^{*}\mu \rangle = \langle \Psi\mathbf{1}_{A}, \mu \rangle
    \end{align}

    Expanding $\Psi\mathbf{1}_{A}$ using the definition of the transition operator:
    \begin{align}
        (\Psi\mathbf{1}_{A})(s,\boldsymbol{b},\boldsymbol{\theta}) & = \sum_{a^i}\pi^{i}(a^{i}|b^{i}; \theta^{i}) \sum_{a^{\neg i}}\pi^{\neg i}(a^{\neg i}|b^{\neg i}; \theta^{\neg i}) \sum_{s'}T(s'|s, \boldsymbol{a})                                                                                                                            \\
                                                                   & \quad \sum_{\boldsymbol{o}'}\boldsymbol{O}(\boldsymbol{o}'|s') \sum_{\boldsymbol{\theta}'}\boldsymbol{U}(\boldsymbol{\theta}'|\boldsymbol{\theta}, \boldsymbol{\tau}) \mathbf{1}_{A}(s', \text{update}(\boldsymbol{b}, \boldsymbol{a}, \boldsymbol{o}'), \boldsymbol{\theta}')
    \end{align}

    Substituting this back and using the definition of inner product:
    \begin{align}
        (\Psi^{*}\mu)(A) & = \int_{S \times \boldsymbol{B} \times \boldsymbol{\Theta}}(\Psi\mathbf{1}_{A})(s,\boldsymbol{b},\boldsymbol{\theta}) \, d\mu(s, \boldsymbol{b}, \boldsymbol{\theta})                                                                                                                                                          \\
                         & = \int_{S \times \boldsymbol{B} \times \boldsymbol{\Theta}}\sum_{a^i}\pi^{i}(a^{i}|b^{i}; \theta^{i}) \sum_{a^{\neg i}}\pi^{\neg i}(a^{\neg i}|b^{\neg i}; \theta^{\neg i}) \sum_{s'}T(s'|s, \boldsymbol{a})                                                                                                                   \\
                         & \quad \sum_{\boldsymbol{o}'}\boldsymbol{O}(\boldsymbol{o}'|s') \sum_{\boldsymbol{\theta}'}\boldsymbol{U}(\boldsymbol{\theta}'|\boldsymbol{\theta}, \boldsymbol{\tau}) \mathbf{1}_{A}(s', \text{update}(\boldsymbol{b}, \boldsymbol{a}, \boldsymbol{o}'), \boldsymbol{\theta}') \, d\mu(s, \boldsymbol{b}, \boldsymbol{\theta})
    \end{align}
\end{proof}

\subsection{Contraction and Propagation}

The properties established above lead to two crucial results: the contraction
property of the discounted transition operator and the propagation of
distributions through the adjoint.

\begin{lemma}[Contraction Mapping Property]
    For any discount factor $\gamma \in [0,1)$, the operator $\gamma \Psi$ is a
    contraction mapping on the Banach space of bounded functions on
    $S \times \boldsymbol{B}\times \boldsymbol{\Theta}$ equipped with the supremum
    norm. Specifically, for any two functions
    $f, g \in \mathcal{B}(S \times \boldsymbol{B}\times \boldsymbol{\Theta})$:
    \begin{align}
        \|\gamma \Psi f - \gamma \Psi g\|_{\infty}\leq \gamma\|f - g\|_{\infty}
    \end{align}
\end{lemma}

\begin{proof}
    Using the linearity of $\Psi$ and its bound preservation property:
    \begin{align}
        \|\gamma \Psi f - \gamma \Psi g\|_{\infty} & = \gamma\|\Psi(f - g)\|_{\infty} \\
                                                   & \leq \gamma\|f - g\|_{\infty}
    \end{align}

    Since $\gamma < 1$, $\gamma \Psi$ is a contraction mapping with contraction
    factor $\gamma$.
\end{proof}

This contraction property leads directly to the invertibility of
$I - \gamma\Psi$, a result that is fundamental for characterizing value functions
in discounted settings.

\begin{lemma}[Invertibility of $I - \gamma \Psi$]
    For any discount factor $\gamma \in [0,1)$, the operator $I - \gamma \Psi$ is
    invertible, and its inverse can be represented as a Neumann series:
    \begin{align}
        (I - \gamma \Psi)^{-1}= \sum_{t=0}^{\infty}\gamma^{t} \Psi^{t}
    \end{align}

    which converges absolutely in the operator norm.
\end{lemma}

\begin{proof}
    By the Banach fixed-point theorem, if $T$ is a contraction mapping on a Banach
    space, then $I - T$ is invertible, with inverse given by the Neumann series
    $\sum_{k=0}^{\infty}T^{k}$. Since $\gamma \Psi$ is a contraction mapping, we
    have:

    \begin{align}
        (I - \gamma \Psi)^{-1} & = \sum_{t=0}^{\infty}(\gamma \Psi)^{t} = \sum_{t=0}^{\infty}\gamma^{t} \Psi^{t}
    \end{align}

    The absolute convergence follows from:
    \begin{align}
        \left\|\sum_{t=0}^{\infty}\gamma^{t} \Psi^{t}\right\|_{\infty} & \leq \sum_{t=0}^{\infty}\gamma^{t} \|\Psi^{t}\|_{\infty}\leq \sum_{t=0}^{\infty}\gamma^{t} = \frac{1}{1-\gamma}< \infty
    \end{align}

    where we use the bound preservation property to establish that $\|\Psi^{t}\|_{\infty}
        \leq 1$ for all $t$.
\end{proof}

Finally, we establish how distributions evolve in the system through the adjoint
operator, formalizing the propagation of probability measures.

\begin{lemma}[Distribution Propagation]
    If $\mu_{t}$ represents the distribution of the joint process $(s_{t}, \boldsymbol
        {b}_{t}, \boldsymbol{\theta}_{t})$ at time $t$, then the distribution at time
    $t+1$ is given by $\mu_{t+1}= \Psi^{*}\mu_{t}$.
\end{lemma}

\begin{proof}
    For any measurable function $f \in \mathcal{B}(S \times \boldsymbol{B}\times \boldsymbol
        {\Theta})$:
    \begin{align}
        \langle f, \mu_{t+1}\rangle & = \mathbb{E}[f(s_{t+1}, \boldsymbol{b}_{t+1}, \boldsymbol{\theta}_{t+1})]                                                                     \\
                                    & = \mathbb{E}[\mathbb{E}[f(s_{t+1}, \boldsymbol{b}_{t+1}, \boldsymbol{\theta}_{t+1}) \mid s_{t}, \boldsymbol{b}_{t}, \boldsymbol{\theta}_{t}]] \\
                                    & = \mathbb{E}[(\Psi f)(s_{t}, \boldsymbol{b}_{t}, \boldsymbol{\theta}_{t})]                                                                    \\
                                    & = \langle \Psi f, \mu_{t} \rangle = \langle f, \Psi^{*}\mu_{t} \rangle
    \end{align}

    Since this holds for any measurable function $f$, we have
    $\mu_{t+1}= \Psi^{*}\mu_{t}$ by the uniqueness of the representing measure.
\end{proof}

The concepts introduced in this section—the transition operator, its adjoint, and
their properties—provide the mathematical foundation for analyzing discounted returns
in Partially Observable Active Markov Games. The transition operator $\Psi$
captures how function expectations evolve forward in time, while its adjoint $\Psi
    ^{*}$ describes how probability distributions propagate forward in time. This
duality is fundamental to establishing the connection between value-based and
distribution-based perspectives in reinforcement learning.

In the next section, we will use these tools to formulate Bellman equations and derive
policy gradients for maximizing discounted returns in this complex multi-agent setting.
\section{Bellman Equations and Value Functions}
In this section, we establish the fundamental recursive relationships that characterize
the discounted value functions in Partially Observable Active Markov Games.
These relationships form the basis for our policy gradient derivation and
practical algorithms for optimizing agent behavior. We begin by confirming the well-definedness
of value functions, then establish the Bellman equations, and finally derive the
gradient expressions necessary for policy optimization.
\subsection{Well-Definedness of Value Functions}
Before deriving recursive relationships, we first establish that the value functions
are well-defined mathematical objects under our assumptions of bounded rewards and
discount factors less than 1.
\begin{lemma}[Well-Defined Value Functions]
    Under the assumptions of bounded rewards and $\gamma < 1$, the discounted
    value functions $v^{i}_{\theta^i}(s, \boldsymbol{b}, \boldsymbol{\theta})$ and
    action-value functions
    $q^{i}_{\theta^i}(s, \boldsymbol{b}, \boldsymbol{\theta}, \boldsymbol{a})$ are
    well-defined and finite for all states, belief states, policy parameters, and actions.
\end{lemma}
\begin{proof}
    Assuming rewards are bounded such that $|R^{i}(s, \boldsymbol{a})| \leq R_{\max}$
    for all $s \in S$ and $\boldsymbol{a}\in A$, we have:
    \begin{align}
        |v^{i}_{\theta^i}(s, \boldsymbol{b}, \boldsymbol{\theta})| & = \left| \mathbb{E}\left[ \sum_{t=0}^{\infty}\gamma^{t} R^{i}(s_{t}, \boldsymbol{a}_{t}) \bigg| s_{0}=s, \boldsymbol{b}_{0}=\boldsymbol{b}, \boldsymbol{\theta}_{0}=\boldsymbol{\theta}, \pi \right] \right| \\
                                                                   & \leq \mathbb{E}\left[ \sum_{t=0}^{\infty}\gamma^{t} |R^{i}(s_{t}, \boldsymbol{a}_{t})| \bigg| s_{0}=s, \boldsymbol{b}_{0}=\boldsymbol{b}, \boldsymbol{\theta}_{0}=\boldsymbol{\theta}, \pi \right]           \\
                                                                   & \leq R_{\max}\sum_{t=0}^{\infty}\gamma^{t} = \frac{R_{\max}}{1-\gamma}< \infty
    \end{align}

    Similarly, for the action-value function:
    \begin{align}
        |q^{i}_{\theta^i}(s, \boldsymbol{b}, \boldsymbol{\theta}, \boldsymbol{a})| & = \left| R^{i}(s, \boldsymbol{a}) + \gamma \mathbb{E}[v^{i}_{\theta^i}(s', \boldsymbol{b}', \boldsymbol{\theta}')] \right| \\
                                                                                   & \leq |R^{i}(s, \boldsymbol{a})| + \gamma \mathbb{E}[|v^{i}_{\theta^i}(s', \boldsymbol{b}', \boldsymbol{\theta}')|]         \\
                                                                                   & \leq R_{\max}+ \gamma \frac{R_{\max}}{1-\gamma}= \frac{R_{\max}}{1-\gamma}< \infty
    \end{align}

    Thus, both value functions are well-defined and bounded.
\end{proof}This lemma ensures that our subsequent derivations involving value
functions are mathematically sound. The boundedness property is particularly important
when we consider expectations and derivatives of these functions.
\subsection{Bellman Equations for Discounted Value Functions}
Next, we establish the recursive relationships between value functions, which
are fundamental to dynamic programming approaches.
\begin{lemma}[Bellman Equation for Discounted Value Functions]
    In a Partially Observable Active Markov Game, the discounted value function
    $v^{i}_{\theta^i}$ satisfies the Bellman equation:
    \begin{align}
        v^{i}_{\theta^i}(s, \boldsymbol{b}, \boldsymbol{\theta}) & = \sum_{a^i}\pi^{i}(a^{i}|b^{i}; \theta^{i}) \sum_{a^{\neg i}}\pi^{\neg i}(a^{\neg i}|b^{\neg i}; \theta^{\neg i}) q^{i}_{\theta^i}(s, \boldsymbol{b}, \boldsymbol{\theta}, \boldsymbol{a})
    \end{align}

    where the action-value function $q^{i}_{\theta^i}$ is defined as:
    \begin{align}
        q^{i}_{\theta^i}(s, \boldsymbol{b}, \boldsymbol{\theta}, \boldsymbol{a}) & = R^{i}(s, \boldsymbol{a}) + \gamma \sum_{s'}T(s'|s, \boldsymbol{a}) \sum_{\boldsymbol{o}'}\boldsymbol{O}(\boldsymbol{o}'|s') \nonumber                                                                                 \\
                                                                                 & \quad \sum_{\boldsymbol{\theta}'}\boldsymbol{U}(\boldsymbol{\theta}'|\boldsymbol{\theta}, \boldsymbol{\tau}) v^{i}_{\theta^i}(s', \text{update}(\boldsymbol{b}, \boldsymbol{a}, \boldsymbol{o}'), \boldsymbol{\theta}')
    \end{align}
\end{lemma}
\begin{proof}
    By definition, the discounted value function is:
    \begin{align}
        v^{i}_{\theta^i}(s, \boldsymbol{b}, \boldsymbol{\theta}) & = \mathbb{E}\left[ \sum_{t=0}^{\infty}\gamma^{t} R^{i}(s_{t}, \boldsymbol{a}_{t}) \bigg| s_{0}=s, \boldsymbol{b}_{0}=\boldsymbol{b}, \boldsymbol{\theta}_{0}=\boldsymbol{\theta}, \pi \right]
    \end{align}

    We can decompose this into the immediate reward and the future discounted
    returns:
    \begin{align}
        v^{i}_{\theta^i}(s, \boldsymbol{b}, \boldsymbol{\theta}) & = \mathbb{E}\left[ R^{i}(s_{0}, \boldsymbol{a}_{0}) + \gamma \sum_{t=1}^{\infty}\gamma^{t-1}R^{i}(s_{t}, \boldsymbol{a}_{t}) \bigg| s_{0}=s, \boldsymbol{b}_{0}=\boldsymbol{b}, \boldsymbol{\theta}_{0}=\boldsymbol{\theta}, \pi \right] \\
                                                                 & = \mathbb{E}\left[ R^{i}(s_{0}, \boldsymbol{a}_{0}) \bigg| s_{0}=s, \boldsymbol{b}_{0}=\boldsymbol{b}, \boldsymbol{\theta}_{0}=\boldsymbol{\theta}, \pi \right] \nonumber                                                                \\
                                                                 & \quad + \gamma \mathbb{E}\left[ \sum_{t=1}^{\infty}\gamma^{t-1}R^{i}(s_{t}, \boldsymbol{a}_{t}) \bigg| s_{0}=s, \boldsymbol{b}_{0}=\boldsymbol{b}, \boldsymbol{\theta}_{0}=\boldsymbol{\theta}, \pi \right]
    \end{align}

    For the first term, we have:
    \begin{align}
        \mathbb{E}\left[ R^{i}(s_{0}, \boldsymbol{a}_{0}) \bigg| s_{0}=s, \boldsymbol{b}_{0}=\boldsymbol{b}, \boldsymbol{\theta}_{0}=\boldsymbol{\theta}, \pi \right] & = \sum_{a^i}\pi^{i}(a^{i}|b^{i}; \theta^{i}) \sum_{a^{\neg i}}\pi^{\neg i}(a^{\neg i}|b^{\neg i}; \theta^{\neg i}) R^{i}(s, \boldsymbol{a})
    \end{align}

    For the second term, by the law of total expectation:
    \begin{align}
         & \mathbb{E}\left[ \sum_{t=1}^{\infty}\gamma^{t-1}R^{i}(s_{t}, \boldsymbol{a}_{t}) \bigg| s_{0}=s, \boldsymbol{b}_{0}=\boldsymbol{b}, \boldsymbol{\theta}_{0}=\boldsymbol{\theta}, \pi \right]                                                                                     \\
         & = \sum_{a^i}\pi^{i}(a^{i}|b^{i}; \theta^{i}) \sum_{a^{\neg i}}\pi^{\neg i}(a^{\neg i}|b^{\neg i}; \theta^{\neg i}) \sum_{s'}T(s'|s, \boldsymbol{a}) \nonumber                                                                                                                    \\
         & \quad \sum_{\boldsymbol{o}'}\boldsymbol{O}(\boldsymbol{o}'|s') \sum_{\boldsymbol{\theta}'}\boldsymbol{U}(\boldsymbol{\theta}'|\boldsymbol{\theta}, \boldsymbol{\tau}) v^{i}_{\theta^i}(s', \text{update}(\boldsymbol{b}, \boldsymbol{a}, \boldsymbol{o}'), \boldsymbol{\theta}')
    \end{align}

    Combining these two terms and factoring out common terms:
    \begin{align}
        v^{i}_{\theta^i}(s, \boldsymbol{b}, \boldsymbol{\theta}) & = \sum_{a^i}\pi^{i}(a^{i}|b^{i}; \theta^{i}) \sum_{a^{\neg i}}\pi^{\neg i}(a^{\neg i}|b^{\neg i}; \theta^{\neg i}) \Bigg[ R^{i}(s, \boldsymbol{a}) \nonumber                                                                   \\
                                                                 & \quad + \gamma \sum_{s'}T(s'|s, \boldsymbol{a}) \sum_{\boldsymbol{o}'}\boldsymbol{O}(\boldsymbol{o}'|s') \nonumber                                                                                                             \\
                                                                 & \quad \sum_{\boldsymbol{\theta}'}\boldsymbol{U}(\boldsymbol{\theta}'|\boldsymbol{\theta}, \boldsymbol{\tau}) v^{i}_{\theta^i}(s', \text{update}(\boldsymbol{b}, \boldsymbol{a}, \boldsymbol{o}'), \boldsymbol{\theta}') \Bigg] \\
                                                                 & = \sum_{a^i}\pi^{i}(a^{i}|b^{i}; \theta^{i}) \sum_{a^{\neg i}}\pi^{\neg i}(a^{\neg i}|b^{\neg i}; \theta^{\neg i}) q^{i}_{\theta^i}(s, \boldsymbol{b}, \boldsymbol{\theta}, \boldsymbol{a})
    \end{align}

    where we define the action-value function $q^{i}_{\theta^i}$ as specified.
\end{proof}The Bellman equation provides a recursive characterization of the
value function that forms the basis for dynamic programming algorithms. This equation
captures how current actions influence both immediate rewards and future value through
their effects on the environment state, belief states, and policy parameters of other
agents. The key distinction from standard Bellman equations is the inclusion of
belief updates and policy parameter dynamics, which reflects the complex dependencies
in partially observable multi-agent settings.
\subsection{Policy Gradient with Respect to Value Function}
Having established the recursive relationship for value functions, we now derive
how the value function gradient depends on policy parameters, which is essential
for policy optimization.
\begin{lemma}[Policy Gradient with Respect to Value Function]
    The gradient of the value function with respect to policy parameters $\theta^{i}$
    can be expressed as:
    \begin{align}
        \nabla_{\theta^i}v^{i}_{\theta^i}(s, \boldsymbol{b}, \boldsymbol{\theta}) & = (I - \gamma \Psi)^{-1}\Bigg[\sum_{a^i}\nabla_{\theta^i}\pi^{i}(a^{i}|b^{i}; \theta^{i}) \nonumber                                                        \\
                                                                                  & \quad \sum_{a^{\neg i}}\pi^{\neg i}(a^{\neg i}|b^{\neg i}; \theta^{\neg i}) q^{i}_{\theta^i}(s, \boldsymbol{b}, \boldsymbol{\theta}, \boldsymbol{a})\Bigg]
    \end{align}

    where $\Psi$ is the transition operator defined in the previous section.
\end{lemma}
\begin{proof}
    Taking the gradient of the Bellman equation with respect to $\theta^{i}$:
    \begin{align}
        \nabla_{\theta^i}v^{i}_{\theta^i}(s, \boldsymbol{b}, \boldsymbol{\theta}) & = \nabla_{\theta^i}\left[ \sum_{a^i}\pi^{i}(a^{i}|b^{i}; \theta^{i}) \sum_{a^{\neg i}}\pi^{\neg i}(a^{\neg i}|b^{\neg i}; \theta^{\neg i}) q^{i}_{\theta^i}(s, \boldsymbol{b}, \boldsymbol{\theta}, \boldsymbol{a}) \right]
    \end{align}

    Applying the product rule:
    \begin{align}
        \nabla_{\theta^i}v^{i}_{\theta^i}(s, \boldsymbol{b}, \boldsymbol{\theta}) & = \sum_{a^i}\nabla_{\theta^i}\pi^{i}(a^{i}|b^{i}; \theta^{i}) \sum_{a^{\neg i}}\pi^{\neg i}(a^{\neg i}|b^{\neg i}; \theta^{\neg i}) q^{i}_{\theta^i}(s, \boldsymbol{b}, \boldsymbol{\theta}, \boldsymbol{a}) \nonumber \\
                                                                                  & \quad + \sum_{a^i}\pi^{i}(a^{i}|b^{i}; \theta^{i}) \sum_{a^{\neg i}}\pi^{\neg i}(a^{\neg i}|b^{\neg i}; \theta^{\neg i}) \nabla_{\theta^i}q^{i}_{\theta^i}(s, \boldsymbol{b}, \boldsymbol{\theta}, \boldsymbol{a})
    \end{align}

    For the gradient of the action-value function:
    \begin{align}
        \nabla_{\theta^i}q^{i}_{\theta^i}(s, \boldsymbol{b}, \boldsymbol{\theta}, \boldsymbol{a}) & = \nabla_{\theta^i}\Bigg[ R^{i}(s, \boldsymbol{a}) + \gamma \sum_{s'}T(s'|s, \boldsymbol{a}) \sum_{\boldsymbol{o}'}\boldsymbol{O}(\boldsymbol{o}'|s') \nonumber                                                                \\
                                                                                                  & \quad \sum_{\boldsymbol{\theta}'}\boldsymbol{U}(\boldsymbol{\theta}'|\boldsymbol{\theta}, \boldsymbol{\tau}) v^{i}_{\theta^i}(s', \text{update}(\boldsymbol{b}, \boldsymbol{a}, \boldsymbol{o}'), \boldsymbol{\theta}') \Bigg]
    \end{align}

    Since $R^{i}(s, \boldsymbol{a})$, $T(s'|s, \boldsymbol{a})$, $\boldsymbol{O}(\boldsymbol
        {o}'|s')$, and
    $\boldsymbol{U}(\boldsymbol{\theta}'|\boldsymbol{\theta}, \boldsymbol{\tau})$
    do not depend directly on $\theta^{i}$:
    \begin{align}
        \nabla_{\theta^i}q^{i}_{\theta^i}(s, \boldsymbol{b}, \boldsymbol{\theta}, \boldsymbol{a}) & = \gamma \sum_{s'}T(s'|s, \boldsymbol{a}) \sum_{\boldsymbol{o}'}\boldsymbol{O}(\boldsymbol{o}'|s') \nonumber                                                                                                                             \\
                                                                                                  & \quad \sum_{\boldsymbol{\theta}'}\boldsymbol{U}(\boldsymbol{\theta}'|\boldsymbol{\theta}, \boldsymbol{\tau}) \nabla_{\theta^i}v^{i}_{\theta^i}(s', \text{update}(\boldsymbol{b}, \boldsymbol{a}, \boldsymbol{o}'), \boldsymbol{\theta}')
    \end{align}

    Using the transition operator $\Psi$, we can rewrite:
    \begin{align}
        \nabla_{\theta^i}v^{i}_{\theta^i}(s, \boldsymbol{b}, \boldsymbol{\theta}) & = \sum_{a^i}\nabla_{\theta^i}\pi^{i}(a^{i}|b^{i}; \theta^{i}) \sum_{a^{\neg i}}\pi^{\neg i}(a^{\neg i}|b^{\neg i}; \theta^{\neg i}) q^{i}_{\theta^i}(s, \boldsymbol{b}, \boldsymbol{\theta}, \boldsymbol{a}) \nonumber \\
                                                                                  & \quad + \gamma \Psi[\nabla_{\theta^i}v^{i}_{\theta^i}](s, \boldsymbol{b}, \boldsymbol{\theta})
    \end{align}

    Rearranging:
    \begin{align}
        (I - \gamma \Psi)[\nabla_{\theta^i}v^{i}_{\theta^i}](s, \boldsymbol{b}, \boldsymbol{\theta}) & = \sum_{a^i}\nabla_{\theta^i}\pi^{i}(a^{i}|b^{i}; \theta^{i}) \nonumber                                                                              \\
                                                                                                     & \quad \sum_{a^{\neg i}}\pi^{\neg i}(a^{\neg i}|b^{\neg i}; \theta^{\neg i}) q^{i}_{\theta^i}(s, \boldsymbol{b}, \boldsymbol{\theta}, \boldsymbol{a})
    \end{align}

    By the lemma on the invertibility of $(I - \gamma \Psi)$, we have:
    \begin{align}
        \nabla_{\theta^i}v^{i}_{\theta^i}(s, \boldsymbol{b}, \boldsymbol{\theta}) & = (I - \gamma \Psi)^{-1}\Bigg[\sum_{a^i}\nabla_{\theta^i}\pi^{i}(a^{i}|b^{i}; \theta^{i}) \nonumber                                                        \\
                                                                                  & \quad \sum_{a^{\neg i}}\pi^{\neg i}(a^{\neg i}|b^{\neg i}; \theta^{\neg i}) q^{i}_{\theta^i}(s, \boldsymbol{b}, \boldsymbol{\theta}, \boldsymbol{a})\Bigg] \\
                                                                                  & = \sum_{t=0}^{\infty}\gamma^{t} \Psi^{t}\Bigg[\sum_{a^i}\nabla_{\theta^i}\pi^{i}(a^{i}|b^{i}; \theta^{i}) \nonumber                                        \\
                                                                                  & \quad \sum_{a^{\neg i}}\pi^{\neg i}(a^{\neg i}|b^{\neg i}; \theta^{\neg i}) q^{i}_{\theta^i}(s, \boldsymbol{b}, \boldsymbol{\theta}, \boldsymbol{a})\Bigg]
    \end{align}
\end{proof}This lemma expresses the value function gradient in terms of an
infinite sum of expected Q-values weighted by policy gradients and propagated
through the transition operator. The operator $(I - \gamma \Psi)^{-1}$ captures how
immediate changes in policy parameters propagate through future timesteps. This
form of the gradient is central to our discounted policy gradient theorem, which
we will develop in the next section. The inverse operator can be represented as
a Neumann series $\sum_{t=0}^{\infty}\gamma^{t} \Psi^{t}$, which has a natural interpretation
as the accumulated effect of policy changes over an infinite horizon, weighted
by the discount factor.
\section{Discounted Visitation Measure}
The infinite sum formulation derived in the previous sections, while
mathematically sound, is not directly amenable to practical computation. In this
section, we reformulate these expressions in terms of a probability distribution,
which allows for more efficient estimation through sampling-based approaches. This
reformulation is central to our policy gradient theorem for discounted returns.
\subsection{Definition and Properties}
\begin{definition}[Discounted Visitation Measure]
    For a Markov process governed by a transition operator $\Psi$ and its adjoint $\Psi
        ^{*}$, with an initial distribution $\mu_{0}$ over the joint state-belief-policy
    space, the discounted visitation measure $d^{\pi}_{\mu_0}$ is defined as:
    \begin{equation}
        d^{\pi}_{\mu_0}:= (1-\gamma) \sum_{t=0}^{\infty}\gamma^{t} \mu_{t}
    \end{equation}

    where $\mu_{t}$ is the distribution at time $t$ when starting from $\mu_{0}$,
    and $\gamma \in [0, 1)$ is the discount factor.
\end{definition}This measure represents the expected discounted frequency with which
different states, belief states, and policy parameters are visited when following
policy $\pi$ starting from the initial distribution $\mu_{0}$. The normalization
factor $(1-\gamma)$ ensures that $d^{\pi}_{\mu_0}$ is a proper probability
distribution, summing to 1 across the entire state-belief-policy space. The discounted
visitation measure plays a central role in reinforcement learning theory,
particularly in policy gradient methods. It provides a natural weighting of states
according to their importance for the discounted objective, as states that are
visited more frequently or earlier in trajectories receive higher weight. This concept
generalizes the stationary distribution used in average-reward settings to the discounted
return framework.
\begin{lemma}[Evolution of Discounted Visitation]
    The discounted visitation measure $d^{\pi}_{\mu_0}$ can be expressed in terms of
    the adjoint operator as:
    \begin{equation}
        d^{\pi}_{\mu_0}= (1-\gamma) \sum_{t=0}^{\infty}\gamma^{t} (\Psi^{*})^{t} \mu_{0}
        = (1-\gamma)(I - \gamma\Psi^{*})^{-1}\mu_{0}
    \end{equation}

    where $\mu_{0}$ is the initial distribution.
\end{lemma}
\begin{proof}
    By the propagation lemma established earlier,
    $\mu_{t} = (\Psi^{*})^{t}\mu_{0}$ represents the distribution at time $t$ when
    starting from $\mu_{0}$. The discounted visitation measure weights these distributions
    by $\gamma^{t}$ and normalizes by $(1-\gamma)$ to ensure it's a proper probability
    measure:
    \begin{align}
        d^{\pi}_{\mu_0} & = (1-\gamma) \sum_{t=0}^{\infty}\gamma^{t} \mu_{t}                \\
                        & = (1-\gamma) \sum_{t=0}^{\infty}\gamma^{t} (\Psi^{*})^{t} \mu_{0}
    \end{align}

    The second equality follows from the Neumann series expansion:
    \begin{align}
        (I - \gamma\Psi^{*})^{-1} & = \sum_{t=0}^{\infty}(\gamma\Psi^{*})^{t} = \sum_{t=0}^{\infty}\gamma^{t} (\Psi^{*})^{t}
    \end{align}

    which converges for $\gamma < 1$ since $\Psi^{*}$ is a non-expansion operator in
    the total variation norm.
\end{proof}This establishes the essential connection between the adjoint operator
and the discounted visitation measure, which is central to the policy gradient
theorem. The measure $d^{\pi}_{\mu_0}$ represents the normalized expected discounted
time spent in each state-belief-policy configuration, weighted according to the
discount factor $\gamma$. This connection allows us to transform expressions involving
$(I - \gamma\Psi)^{-1}$ in the value function to expectations with respect to $d^{\pi}
        _{\mu_0}$, which is key for deriving practical policy gradient algorithms.In
practical implementations, the discounted visitation measure is rarely computed explicitly
due to the prohibitive computational cost. Instead, sampling-based approaches are
used to estimate expectations with respect to this measure. The policy gradient
theorem leverages this measure to derive update rules that can be efficiently
implemented using trajectory samples collected from the environment.

\subsection{Existence and Uniqueness}
To provide a rigorous mathematical foundation for our policy gradient theorem,
we establish the existence and uniqueness of the discounted visitation measure.
\begin{lemma}[Existence of the Discounted Visitation Measure]
    Let $\mathcal{M}(S \times \boldsymbol{B}\times \boldsymbol{\Theta})$ be the space
    of finite measures on the joint state-belief-policy space, equipped with the
    total variation norm. For any initial measure $\mu_{0} \in \mathcal{M}(S \times
        \boldsymbol{B}\times \boldsymbol{\Theta})$ and discount factor $\gamma \in [0,1
        )$, the discounted state-visitation measure
    \begin{align}
        d^{\pi}_{\mu_0}= (1-\gamma) \sum_{t=0}^{\infty}\gamma^{t} (\Psi^{*})^{t} \mu_{0}
    \end{align}

    exists and is a well-defined measure in $\mathcal{M}(S \times \boldsymbol{B}\times
        \boldsymbol{\Theta})$.
\end{lemma}
\begin{proof}
    To establish the existence of $d^{\pi}_{\mu_0}$, we need to prove that the
    infinite sum $(1-\gamma) \sum_{t=0}^{\infty}\gamma^{t} (\Psi^{*})^{t} \mu_{0}$
    converges to a well-defined measure. First, let's recall that
    $\mathcal{M}(S \times \boldsymbol{B}\times \boldsymbol{\Theta})$ equipped with
    the total variation norm
    $\|\mu\|_{TV}= \sup_{|f| \leq 1}\left|\int f d\mu\right|$ is a Banach space. Now,
    we need to show that the series
    $\sum_{t=0}^{\infty}\gamma^{t} (\Psi^{*})^{t} \mu_{0}$ converges in this space.
    For any $t \geq 0$, $(\Psi^{*})^{t} \mu_{0}$ represents the measure over the
    joint state-belief-policy space after $t$ transitions, starting from the initial
    measure $\mu_{0}$. Since $\Psi^{*}$ is a Markov operator (the adjoint of a Markov
    transition kernel), it preserves the total mass of a measure: if $\mu_{0}$ is
    a probability measure (i.e., $\mu_{0}(S \times \boldsymbol{B}\times \boldsymbol
        {\Theta}) = 1$), then so is $(\Psi^{*})^{t} \mu_{0}$ for all $t \geq 0$.
    Therefore, $\|(\Psi^{*})^{t} \mu_{0}\|_{TV}= \mu_{0}(S \times \boldsymbol{B}\times
        \boldsymbol{\Theta})$ for all $t \geq 0$. For a general finite measure
    $\mu_{0}$ with total mass
    $M = \mu_{0}(S \times \boldsymbol{B}\times \boldsymbol{\Theta})$, we have $\|(\Psi
        ^{*})^{t} \mu_{0}\|_{TV}= M$ for all $t \geq 0$. Now, consider the partial sum:
    \begin{align}
        S_{n} = (1-\gamma) \sum_{t=0}^{n}\gamma^{t} (\Psi^{*})^{t} \mu_{0}
    \end{align}

    For $n < m$, the difference between two partial sums is:
    \begin{align}
        \|S_{m} - S_{n}\|_{TV} & = \left\| (1-\gamma) \sum_{t=n+1}^{m}\gamma^{t} (\Psi^{*})^{t} \mu_{0} \right\|_{TV} \\
                               & \leq (1-\gamma) \sum_{t=n+1}^{m}\gamma^{t} \|(\Psi^{*})^{t} \mu_{0}\|_{TV}           \\
                               & = (1-\gamma) M \sum_{t=n+1}^{m}\gamma^{t}                                            \\
                               & = (1-\gamma) M (\gamma^{n+1}+ \gamma^{n+2}+ \ldots + \gamma^{m})                     \\
                               & = (1-\gamma) M \gamma^{n+1}\frac{1 - \gamma^{m-n}}{1 - \gamma}                       \\
                               & = M \gamma^{n+1}(1 - \gamma^{m-n})
    \end{align}

    As $n \to \infty$, $\gamma^{n+1}\to 0$ (since $\gamma < 1$), which means that
    the sequence of partial sums $\{S_{n}\}$ is Cauchy in the Banach space
    $\mathcal{M}(S \times \boldsymbol{B}\times \boldsymbol{\Theta})$. By the completeness
    of this space, the sequence converges to a measure which we denote as
    $d^{\pi}_{\mu_0}$. Moreover, the total mass of $d^{\pi}_{\mu_0}$ is:
    \begin{align}
        d^{\pi}_{\mu_0}(S \times \boldsymbol{B}\times \boldsymbol{\Theta}) & = (1-\gamma) \sum_{t=0}^{\infty}\gamma^{t} (\Psi^{*})^{t} \mu_{0}(S \times \boldsymbol{B}\times \boldsymbol{\Theta}) \\
                                                                           & = (1-\gamma) M \sum_{t=0}^{\infty}\gamma^{t}                                                                         \\
                                                                           & = (1-\gamma) M \frac{1}{1-\gamma}                                                                                    \\
                                                                           & = M
    \end{align}

    Thus, $d^{\pi}_{\mu_0}$ is a finite measure with the same total mass as $\mu_{0}$.
\end{proof}Having established existence, we now turn to the uniqueness of the discounted
visitation measure.

\begin{lemma}[Uniqueness of the Discounted Visitation Measure]
    The discounted state-visitation measure $d^{\pi}_{\mu_0}$ is the unique solution
    to the functional equation:
    \begin{align}
        d^{\pi}_{\mu_0}= (1-\gamma)\mu_{0} + \gamma \Psi^{*} d^{\pi}_{\mu_0}
    \end{align}

    in the space of finite measures $\mathcal{M}(S \times \boldsymbol{B}\times \boldsymbol
        {\Theta})$.
\end{lemma}
\begin{proof}
    First, we verify that
    $d^{\pi}_{\mu_0}= (1-\gamma) \sum_{t=0}^{\infty}\gamma^{t} (\Psi^{*})^{t} \mu_{0}$
    satisfies the functional equation:

    \begin{align}
        (1-\gamma)\mu_{0} + \gamma \Psi^{*} d^{\pi}_{\mu_0} & = (1-\gamma)\mu_{0} + \gamma \Psi^{*} \left( (1-\gamma) \sum_{t=0}^{\infty}\gamma^{t} (\Psi^{*})^{t} \mu_{0} \right) \\
                                                            & = (1-\gamma)\mu_{0} + \gamma (1-\gamma) \sum_{t=0}^{\infty}\gamma^{t} (\Psi^{*})^{t+1}\mu_{0}                        \\
                                                            & = (1-\gamma)\mu_{0} + (1-\gamma) \sum_{t=1}^{\infty}\gamma^{t} (\Psi^{*})^{t} \mu_{0}                                \\
                                                            & = (1-\gamma) \left( \mu_{0} + \sum_{t=1}^{\infty}\gamma^{t} (\Psi^{*})^{t} \mu_{0} \right)                           \\
                                                            & = (1-\gamma) \sum_{t=0}^{\infty}\gamma^{t} (\Psi^{*})^{t} \mu_{0}                                                    \\
                                                            & = d^{\pi}_{\mu_0}
    \end{align}

    To prove uniqueness, suppose there exists another measure
    $\tilde{d}\in \mathcal{M}(S \times \boldsymbol{B}\times \boldsymbol{\Theta})$
    that satisfies the functional equation:
    \begin{align}
        \tilde{d}= (1-\gamma)\mu_{0} + \gamma \Psi^{*} \tilde{d}
    \end{align}

    Let's consider the operator
    $T: \mathcal{M}(S \times \boldsymbol{B}\times \boldsymbol{\Theta}) \to \mathcal{M}
        (S \times \boldsymbol{B}\times \boldsymbol{\Theta})$
    defined as:
    \begin{align}
        T(\mu) = (1-\gamma)\mu_{0} + \gamma \Psi^{*} \mu
    \end{align}

    For any two measures
    $\mu, \nu \in \mathcal{M}(S \times \boldsymbol{B}\times \boldsymbol{\Theta})$:
    \begin{align}
        \|T(\mu) - T(\nu)\|_{TV} & = \|\gamma \Psi^{*} \mu - \gamma \Psi^{*} \nu\|_{TV} \\
                                 & = \gamma \|\Psi^{*} (\mu - \nu)\|_{TV}               \\
                                 & \leq \gamma \|\mu - \nu\|_{TV}
    \end{align}

    Since $\gamma < 1$, $T$ is a contraction mapping on the Banach space $\mathcal{M}
        (S \times \boldsymbol{B}\times \boldsymbol{\Theta})$ with contraction factor $\gamma$.
    By the Banach fixed-point theorem, $T$ has a unique fixed point. We've already
    shown that $d^{\pi}_{\mu_0}$ is a fixed point of $T$, and by assumption,
    $\tilde{d}$ is also a fixed point. Therefore, $d^{\pi}_{\mu_0}= \tilde{d}$, proving
    the uniqueness of the discounted state-visitation measure.
\end{proof}

\subsection{Connection to Value Function}
Having established the theoretical properties of the discounted visitation
measure, we now connect it to the value function, which will provide important
insights later on.

\begin{lemma}[Value Function as Inner Product]
    For any discount factor $\gamma \in [0,1)$ and initial state $s_{0}$, belief
    state $\boldsymbol{b}_{0}$, and policy parameter $\boldsymbol{\theta}_{0}$,
    the value function can be expressed as:
    \begin{align}
        v^{i}_{\theta^i}(s_{0}, \boldsymbol{b}_{0}, \boldsymbol{\theta}_{0}) & = \frac{1}{1-\gamma}\int_{S \times \boldsymbol{B} \times \boldsymbol{\Theta}}r^{i}(s, \boldsymbol{b}, \boldsymbol{\theta}) d^{\pi}_{\mu_0}(s, \boldsymbol{b}, \boldsymbol{\theta})
    \end{align}

    where $\mu_{0} = \delta_{(s_0, \boldsymbol{b}_0, \boldsymbol{\theta}_0)}$ is
    the Dirac measure concentrated at
    $(s_{0}, \boldsymbol{b}_{0}, \boldsymbol{\theta}_{0})$, and $r^{i}$ is the expected
    immediate reward function:
    \begin{align}
        r^{i}(s, \boldsymbol{b}, \boldsymbol{\theta}) = \sum_{a^i}\pi^{i}(a^{i}|b^{i}; \theta^{i}) \sum_{a^{\neg i}}\pi^{\neg i}(a^{\neg i}|b^{\neg i}; \theta^{\neg i}) R^{i}(s, \boldsymbol{a})
    \end{align}
\end{lemma}
\begin{proof}
    Let $\mu_{0} = \delta_{(s_0, \boldsymbol{b}_0, \boldsymbol{\theta}_0)}$ be the
    Dirac measure concentrated at the point $(s_{0}, \boldsymbol{b}_{0}, \boldsymbol
        {\theta}_{0})$. By definition of the value function:
    \begin{align}
        v^{i}_{\theta^i}(s_{0}, \boldsymbol{b}_{0}, \boldsymbol{\theta}_{0}) & = \mathbb{E}\left[ \sum_{t=0}^{\infty}\gamma^{t} R^{i}(s_{t}, \boldsymbol{a}_{t}) \bigg| s_{0}, \boldsymbol{b}_{0}, \boldsymbol{\theta}_{0}, \pi \right] \\
                                                                             & = \sum_{t=0}^{\infty}\gamma^{t} \mathbb{E}\left[ R^{i}(s_{t}, \boldsymbol{a}_{t}) \bigg| s_{0}, \boldsymbol{b}_{0}, \boldsymbol{\theta}_{0}, \pi \right]
    \end{align}

    Define the expected immediate reward function
    $r^{i}: S \times \boldsymbol{B}\times \boldsymbol{\Theta}\to \mathbb{R}$ as:
    \begin{align}
        r^{i}(s, \boldsymbol{b}, \boldsymbol{\theta}) = \sum_{a^i}\pi^{i}(a^{i}|b^{i}; \theta^{i}) \sum_{a^{\neg i}}\pi^{\neg i}(a^{\neg i}|b^{\neg i}; \theta^{\neg i}) R^{i}(s, \boldsymbol{a})
    \end{align}

    Then:
    \begin{align}
        v^{i}_{\theta^i}(s_{0}, \boldsymbol{b}_{0}, \boldsymbol{\theta}_{0}) & = \sum_{t=0}^{\infty}\gamma^{t} \mathbb{E}\left[ r^{i}(s_{t}, \boldsymbol{b}_{t}, \boldsymbol{\theta}_{t}) \bigg| s_{0}, \boldsymbol{b}_{0}, \boldsymbol{\theta}_{0}, \pi \right]       \\
                                                                             & = \sum_{t=0}^{\infty}\gamma^{t} \int_{S \times \boldsymbol{B} \times \boldsymbol{\Theta}}r^{i}(s, \boldsymbol{b}, \boldsymbol{\theta}) d\mu_{t}(s, \boldsymbol{b}, \boldsymbol{\theta})
    \end{align}

    where $\mu_{t}$ is the distribution over the joint state-belief-policy space
    at time $t$, starting from $\mu_{0}$. From previous lemmas, we know that $\mu_{t}
        = (\Psi^{*})^{t} \mu_{0}$. Using the duality relationship between functions
    and measures:
    \begin{align}
        \int_{S \times \boldsymbol{B} \times \boldsymbol{\Theta}}r^{i}(s, \boldsymbol{b}, \boldsymbol{\theta}) d\mu_{t}(s, \boldsymbol{b}, \boldsymbol{\theta}) & = \int_{S \times \boldsymbol{B} \times \boldsymbol{\Theta}}r^{i}(s, \boldsymbol{b}, \boldsymbol{\theta}) d((\Psi^{*})^{t}\mu_{0})(s, \boldsymbol{b}, \boldsymbol{\theta}) \\
                                                                                                                                                                & = \int_{S \times \boldsymbol{B} \times \boldsymbol{\Theta}}(\Psi^{t} r^{i})(s, \boldsymbol{b}, \boldsymbol{\theta}) d\mu_{0}(s, \boldsymbol{b}, \boldsymbol{\theta})
    \end{align}

    Therefore:
    \begin{align}
        v^{i}_{\theta^i}(s_{0}, \boldsymbol{b}_{0}, \boldsymbol{\theta}_{0}) & = \sum_{t=0}^{\infty}\gamma^{t} \int_{S \times \boldsymbol{B} \times \boldsymbol{\Theta}}(\Psi^{t} r^{i})(s, \boldsymbol{b}, \boldsymbol{\theta}) d\mu_{0}(s, \boldsymbol{b}, \boldsymbol{\theta})              \\
                                                                             & = \int_{S \times \boldsymbol{B} \times \boldsymbol{\Theta}}\left(\sum_{t=0}^{\infty}\gamma^{t} (\Psi^{t} r^{i})(s, \boldsymbol{b}, \boldsymbol{\theta})\right) d\mu_{0}(s, \boldsymbol{b}, \boldsymbol{\theta}) \\
                                                                             & = \int_{S \times \boldsymbol{B} \times \boldsymbol{\Theta}}((I - \gamma \Psi)^{-1}r^{i})(s, \boldsymbol{b}, \boldsymbol{\theta}) d\mu_{0}(s, \boldsymbol{b}, \boldsymbol{\theta})
    \end{align}

    Since $\mu_{0} = \delta_{(s_0, \boldsymbol{b}_0, \boldsymbol{\theta}_0)}$, we have:
    \begin{align}
        v^{i}_{\theta^i}(s_{0}, \boldsymbol{b}_{0}, \boldsymbol{\theta}_{0}) & = ((I - \gamma \Psi)^{-1}r^{i})(s_{0}, \boldsymbol{b}_{0}, \boldsymbol{\theta}_{0})
    \end{align}

    Alternatively, using the duality between $(I - \gamma \Psi)^{-1}$ and
    $(I - \gamma \Psi^{*})^{-1}$:
    \begin{align}
        \int_{S \times \boldsymbol{B} \times \boldsymbol{\Theta}}((I - \gamma \Psi)^{-1}r^{i})(s, \boldsymbol{b}, \boldsymbol{\theta}) d\mu_{0}(s, \boldsymbol{b}, \boldsymbol{\theta}) & = \int_{S \times \boldsymbol{B} \times \boldsymbol{\Theta}}r^{i}(s, \boldsymbol{b}, \boldsymbol{\theta}) d((I - \gamma \Psi^{*})^{-1}\mu_{0})(s, \boldsymbol{b}, \boldsymbol{\theta})
    \end{align}

    From previous lemmas, we know that $d^{\pi}_{\mu_0}= (1-\gamma) (I - \gamma \Psi
        ^{*})^{-1}\mu_{0}$. Therefore:
    \begin{align}
        v^{i}_{\theta^i}(s_{0}, \boldsymbol{b}_{0}, \boldsymbol{\theta}_{0}) & = \int_{S \times \boldsymbol{B} \times \boldsymbol{\Theta}}r^{i}(s, \boldsymbol{b}, \boldsymbol{\theta}) d((I - \gamma \Psi^{*})^{-1}\mu_{0})(s, \boldsymbol{b}, \boldsymbol{\theta}) \\
                                                                             & = \int_{S \times \boldsymbol{B} \times \boldsymbol{\Theta}}r^{i}(s, \boldsymbol{b}, \boldsymbol{\theta}) \frac{d^{\pi}_{\mu_0}(s, \boldsymbol{b}, \boldsymbol{\theta})}{1-\gamma}     \\
                                                                             & = \frac{1}{1-\gamma}\int_{S \times \boldsymbol{B} \times \boldsymbol{\Theta}}r^{i}(s, \boldsymbol{b}, \boldsymbol{\theta}) d^{\pi}_{\mu_0}(s, \boldsymbol{b}, \boldsymbol{\theta})
    \end{align}
\end{proof}This lemma establishes that the value function can be expressed as an
expectation with respect to the discounted visitation measure, with an appropriate
scaling factor. This formulation provides a direct link between the value
function and the distribution of states.

\section{Policy Gradient Theorem}
Having established the mathematical foundations for discounted returns in
partially observable multi-agent settings, we now derive the policy gradient theorem
that forms the basis for practical optimization algorithms. This theorem
provides a principled expression for computing gradients of the expected
discounted return with respect to policy parameters, enabling efficient policy improvement.
The policy gradient theorem connects an agent's policy parameters to its
expected long-term performance through the discounted visitation measure, providing
a mathematically rigorous foundation for optimization.

\begin{theorem}[Partially Observable Active Discounted Return Policy Gradient
        Theorem]
    The gradient of the discounted return objective $J^{i}_{\pi, \gamma}(\theta^{i}) = v
            ^{i}_{\theta^i}(s_{0}, \boldsymbol{b}_{0}, \boldsymbol{\theta}_{0})$ with respect
    to agent $i$'s policy parameters $\theta^{i}$ in a partially observable active
    Markov game setting can be expressed as:
    \begin{align}
        \nabla_{\theta^i}J^{i}_{\pi, \gamma}(\theta^{i}) & = \frac{1}{1-\gamma}\sum_{s, \boldsymbol{b}, \boldsymbol{\theta}}d^{\pi}_{\mu_0}(s, \boldsymbol{b}, \boldsymbol{\theta}) \sum_{a^i}\nabla_{\theta^i}\pi^{i}(a^{i}|b^{i}; \theta^{i}) \nonumber \\
                                                         & \quad \sum_{a^{-i}}\pi^{-i}(a^{-i}|\boldsymbol{b}^{-i}; \theta^{-i}) q^{i}_{\theta^i}(s, \boldsymbol{b}, \boldsymbol{\theta}, \boldsymbol{a})
    \end{align}

    where $d^{\pi}_{\mu_0}(s, \boldsymbol{b}, \boldsymbol{\theta})$ is the discounted
    visitation measure starting from initial distribution $\mu_{0}$, and
    $q^{i}_{\theta^i}(s, \boldsymbol{b}, \boldsymbol{\theta}, \boldsymbol{a})$ is
    the action-value function defined as:
    \begin{align}
        q^{i}_{\theta^i}(s, \boldsymbol{b}, \boldsymbol{\theta}, \boldsymbol{a}) & = R^{i}(s, \boldsymbol{a}) + \gamma \sum_{s'}T(s'|s, \boldsymbol{a}) \sum_{\boldsymbol{o}'}\boldsymbol{O}(\boldsymbol{o}'|s') \nonumber                                                                                 \\
                                                                                 & \quad \sum_{\boldsymbol{\theta}'}\boldsymbol{U}(\boldsymbol{\theta}'|\boldsymbol{\theta}, \boldsymbol{\tau}) v^{i}_{\theta^i}(s', \text{update}(\boldsymbol{b}, \boldsymbol{a}, \boldsymbol{o}'), \boldsymbol{\theta}')
    \end{align}
\end{theorem}We now provide a detailed proof of the policy gradient theorem, building
on the lemmas established in previous sections.
\begin{proof}
    From our previous lemma, we know that the gradient of the value function can be
    expressed as:
    \begin{align}
        \nabla_{\theta^i}v^{i}_{\theta^i}(s, \boldsymbol{b}, \boldsymbol{\theta}) & = (I - \gamma \Psi)^{-1}\left[\sum_{a^i}\nabla_{\theta^i}\pi^{i}(a^{i}|b^{i}; \theta^{i}) \sum_{a^{-i}}\pi^{-i}(a^{-i}|\boldsymbol{b}^{-i}; \theta^{-i}) q^{i}_{\theta^i}(s, \boldsymbol{b}, \boldsymbol{\theta}, \boldsymbol{a})\right]
    \end{align}

    For notational clarity, we define the term inside the brackets as:
    \begin{align}
        g(s, \boldsymbol{b}, \boldsymbol{\theta}) = \sum_{a^i}\nabla_{\theta^i}\pi^{i}(a^{i}|b^{i}; \theta^{i}) \sum_{a^{-i}}\pi^{-i}(a^{-i}|\boldsymbol{b}^{-i}; \theta^{-i}) q^{i}_{\theta^i}(s, \boldsymbol{b}, \boldsymbol{\theta}, \boldsymbol{a})
    \end{align}

    Now we can write:
    \begin{align}
        \nabla_{\theta^i}v^{i}_{\theta^i}(s, \boldsymbol{b}, \boldsymbol{\theta}) & = (I - \gamma \Psi)^{-1}g(s, \boldsymbol{b}, \boldsymbol{\theta})
    \end{align}

    For our specific initial state configuration
    $(s_{0}, \boldsymbol{b}_{0}, \boldsymbol{\theta}_{0})$:
    \begin{align}
        \nabla_{\theta^i}v^{i}_{\theta^i}(s_{0}, \boldsymbol{b}_{0}, \boldsymbol{\theta}_{0}) & = ((I - \gamma \Psi)^{-1}g)(s_{0}, \boldsymbol{b}_{0}, \boldsymbol{\theta}_{0})
    \end{align}

    Let's define
    $\mu_{0} = \delta_{(s_0, \boldsymbol{b}_0, \boldsymbol{\theta}_0)}$, which is the
    Dirac measure concentrated at the initial state. We can view
    $((I - \gamma \Psi)^{-1}g)(s_{0}, \boldsymbol{b}_{0}, \boldsymbol{\theta}_{0})$
    as the integration of the function $(I - \gamma \Psi)^{-1}g$ against this
    Dirac measure:

    \begin{align}
        \nabla_{\theta^i}v^{i}_{\theta^i}(s_{0}, \boldsymbol{b}_{0}, \boldsymbol{\theta}_{0}) & = ((I - \gamma \Psi)^{-1}g)(s_{0}, \boldsymbol{b}_{0}, \boldsymbol{\theta}_{0})                                                                                           \\
                                                                                              & = \int ((I - \gamma \Psi)^{-1}g)(s, \boldsymbol{b}, \boldsymbol{\theta}) d\delta_{(s_0, \boldsymbol{b}_0, \boldsymbol{\theta}_0)}(s, \boldsymbol{b}, \boldsymbol{\theta})
    \end{align}

    Recall the duality principle we established earlier, which states that:
    \begin{align}
        \int ((I - \gamma \Psi)^{-1}f)(x) d\mu(x) = \int f(x) d((I - \gamma \Psi^{*})^{-1}\mu)(x)
    \end{align}

    Allowing us to move the operator from acting on the function $g$ to acting on
    the measure $\mu_{0}$:
    \begin{align}
        \int ((I - \gamma \Psi)^{-1}g)(s, \boldsymbol{b}, \boldsymbol{\theta}) d\mu_{0}(s, \boldsymbol{b}, \boldsymbol{\theta}) = \int g(s, \boldsymbol{b}, \boldsymbol{\theta}) d((I - \gamma \Psi^{*})^{-1}\mu_{0})(s, \boldsymbol{b}, \boldsymbol{\theta})
    \end{align}

    Therefore:
    \begin{align}
        \nabla_{\theta^i}v^{i}_{\theta^i}(s_{0}, \boldsymbol{b}_{0}, \boldsymbol{\theta}_{0}) & = \int g(s, \boldsymbol{b}, \boldsymbol{\theta}) d((I - \gamma \Psi^{*})^{-1}\delta_{(s_0, \boldsymbol{b}_0, \boldsymbol{\theta}_0)})(s, \boldsymbol{b}, \boldsymbol{\theta})
    \end{align}

    From our earlier derivation of the discounted visitation measure, we established
    that:
    \begin{align}
        d^{\pi}_{\mu_0}= (1-\gamma)(I - \gamma \Psi^{*})^{-1}\mu_{0} = (1-\gamma)(I - \gamma \Psi^{*})^{-1}\delta_{(s_0, \boldsymbol{b}_0, \boldsymbol{\theta}_0)}
    \end{align}

    We can now substitute this expression:
    \begin{align}
        \nabla_{\theta^i}v^{i}_{\theta^i}(s_{0}, \boldsymbol{b}_{0}, \boldsymbol{\theta}_{0}) & = \int g(s, \boldsymbol{b}, \boldsymbol{\theta}) d\left(\frac{d^{\pi}_{\mu_0}}{1-\gamma}\right)(s, \boldsymbol{b}, \boldsymbol{\theta}) \\
                                                                                              & = \frac{1}{1-\gamma}\int g(s, \boldsymbol{b}, \boldsymbol{\theta}) d^{\pi}_{\mu_0}(s, \boldsymbol{b}, \boldsymbol{\theta})
    \end{align}

    Substituting the definition of $g(s, \boldsymbol{b}, \boldsymbol{\theta})$ back:
    \begin{align}
        \nabla_{\theta^i}J^{i}_{\pi, \gamma}(\theta^{i}) & = \nabla_{\theta^i}v^{i}_{\theta^i}(s_{0}, \boldsymbol{b}_{0}, \boldsymbol{\theta}_{0})                                                                                                                                                                                                                                                  \\
                                                         & = \frac{1}{1-\gamma}\int_{S \times \boldsymbol{B} \times \boldsymbol{\Theta}}\sum_{a^i}\nabla_{\theta^i}\pi^{i}(a^{i}|b^{i}; \theta^{i}) \sum_{a^{-i}}\pi^{-i}(a^{-i}|\boldsymbol{b}^{-i}; \theta^{-i}) q^{i}_{\theta^i}(s, \boldsymbol{b}, \boldsymbol{\theta}, \boldsymbol{a}) d^{\pi}_{\mu_0}(s, \boldsymbol{b}, \boldsymbol{\theta})
    \end{align}

    For finite state, belief, and policy parameter spaces, this integral can be
    written as a sum:
    \begin{align}
        \nabla_{\theta^i}J^{i}_{\pi, \gamma}(\theta^{i}) & = \frac{1}{1-\gamma}\sum_{s, \boldsymbol{b}, \boldsymbol{\theta}}d^{\pi}_{\mu_0}(s, \boldsymbol{b}, \boldsymbol{\theta}) \sum_{a^i}\nabla_{\theta^i}\pi^{i}(a^{i}|b^{i}; \theta^{i}) \sum_{a^{-i}}\pi^{-i}(a^{-i}|\boldsymbol{b}^{-i}; \theta^{-i}) q^{i}_{\theta^i}(s, \boldsymbol{b}, \boldsymbol{\theta}, \boldsymbol{a})
    \end{align}

    This completes the proof of the Partially Observable Active Discounted Return
    Policy Gradient Theorem.
\end{proof}

The proof reveals a symmetry in the mathematical structure of our results.
\begin{remark}[Symmetry]
    From the Value Function as Inner Product lemma, the value function
    \begin{align}
        v^{i}_{\theta^i}(s_{0}, \boldsymbol{b}_{0}, \boldsymbol{\theta}_{0}) = \frac{1}{1-\gamma}\int_{S \times \boldsymbol{B} \times \boldsymbol{\Theta}}r^{i}(s, \boldsymbol{b}, \boldsymbol{\theta}) d^{\pi}_{\mu_0}(s, \boldsymbol{b}, \boldsymbol{\theta})
    \end{align}

    is an expectation of immediate rewards $r^{i}$ under the discounted visitation
    measure $d^{\pi}_{\mu_0}$. While its gradient
    \begin{align}
        \nabla_{\theta^i}v^{i}_{\theta^i}(s_{0}, \boldsymbol{b}_{0}, \boldsymbol{\theta}_{0}) = \frac{1}{1-\gamma}\int_{S \times \boldsymbol{B} \times \boldsymbol{\Theta}}g(s, \boldsymbol{b}, \boldsymbol{\theta}) d^{\pi}_{\mu_0}(s, \boldsymbol{b}, \boldsymbol{\theta})
    \end{align}

    is an expectation of policy-gradient-weighted action values $g$ under the same
    discounted visitation measure $d^{\pi}_{\mu_0}$.
\end{remark}
This symmetric structure provides not only mathematical elegance but also
practical advantages. It means that the same samples from the discounted visitation
distribution can be used to estimate both values and policy gradients.

The policy gradient theorem provides the mathematical foundation for agents to optimize
their policies in ways that account for both the partial observability of the environment
and the learning dynamics of other agents. This enables sophisticated strategic
behaviors such as information revelation, teaching, and influence that are
essential in social learning contexts.

\chapter{POLARIS Architecture}
\label{appendix:neural_architectures}

This section provides detailed technical specifications of all neural network architectures used in our POLARIS implementation. We describe each component's structure, activation functions, and design considerations to enable reproducibility and thorough understanding of our approach. POLARIS represents a significant advancement over traditional multi-agent reinforcement learning frameworks by employing Graph Neural Networks (GNNs) as the primary architecture for modeling agent interactions and temporal dependencies in partially observable social learning environments.

POLARIS employs a modular architecture consisting of four primary components that work in tandem to enable effective learning in partially observable multi-agent environments. The belief processing module maintains belief states using transformers with attention mechanisms, while the inference learning module models other agents' behavior using Temporal Graph Neural Networks. The policy network maps beliefs and latent states to action decisions, and dual value networks estimate expected returns using a critic architecture. All components operate over the joint space of states, belief states, and policy parameters as defined in our Partially Observable Active Markov Game framework, explicitly modeling how actions influence both environmental dynamics and other agents' learning processes.

\section{Transformer Network}

The belief processing module maintains and updates agents' belief states about the underlying environment state using a transformer-based architecture~\cite{vaswani2017attention}. This component processes sequences of observations and actions to form beliefs about the partially observable environment. The belief processor employs a transformer encoder with an input dimension configurable based on the observation space, using dimension $|S|$ for discrete observation spaces and dimension 1 for continuous observation spaces. The hidden dimension defaults to 256, with 4 attention heads across 2 layers and a dropout rate of 0.1.

For discrete observation spaces, the input signal $\mathbf{o}_t \in \mathbb{R}^{|S|}$ is projected to the hidden dimension through $\mathbf{x}_{\text{proj}} = W_{\text{proj}}\mathbf{o}_t + \mathbf{b}_{\text{proj}}$. For continuous observation spaces with $\mathbf{o}_t \in \mathbb{R}$, a separate projection layer handles the scalar input: $\mathbf{x}_{\text{proj}} = W_{\text{cont}}\mathbf{o}_t + \mathbf{b}_{\text{cont}}$. Positional encodings are added to capture temporal ordering: $\mathbf{x}_{\text{input}} = \mathbf{x}_{\text{proj}} + \mathbf{PE}$, where $\mathbf{PE}$ is a learned positional encoding parameter.

The transformer processes the sequence through multi-head self-attention layers: $\text{MultiHead}(\mathbf{X}) = \text{Concat}(\text{head}_1, \ldots, \text{head}_h)W^O$, where each attention head computes:

\begin{equation}
    \text{head}_i = \text{Attention}(\mathbf{X}W^Q_i, \mathbf{X}W^K_i, \mathbf{X}W^V_i)
\end{equation}

The final belief distribution is computed using appropriate activations based on the observation space: $\mathbf{b}_{\text{dist}} = \text{softmax}(W_{\text{discrete}}\mathbf{h}_{\text{final}} + \mathbf{b}_{\text{discrete}})$ for discrete observation spaces and $\mathbf{b}_{\text{dist}} = \text{sigmoid}(W_{\text{continuous}}\mathbf{h}_{\text{final}} + \mathbf{b}_{\text{continuous}})$ for continuous observation spaces, where $\mathbf{h}_{\text{final}}$ is the output from the transformer layers.

\section{Graph Neural Network}

The core innovation of POLARIS lies in its use of Graph Neural Networks~\citep{hamilton2020graph} to explicitly model the network structure of agent interactions. Unlike traditional approaches that treat other agents as part of the environment, our GNN-based architecture captures the relational structure between agents and enables sophisticated reasoning about network effects in social learning.

The Temporal GNN serves as the inference learning module, enabling agents to predict and model the behavior of other agents over time. This component represents agents and their interactions as a dynamic graph where nodes correspond to agents and edges represent observational relationships. For each agent $i$, the GNN constructs a graph representation where each agent becomes a node with features encoding their current observation and action, connections represent observational relationships based on network topology, and node features consist of the concatenation of belief state and action: $[\mathbf{b}^i_t; a^i_t]$ for the ego agent, $[\mathbf{0}; a^j_t]$ for neighbors. allocation values are directly included in the feature representation.

The GNN employs Graph Attention Networks (GAT)~\citep{veličković2018graph} that compute attention coefficients between connected nodes:

\begin{equation}
    e_{ij} = \text{LeakyReLU}\left(\mathbf{a}^T\left[W\mathbf{h}_i \| W\mathbf{h}_j\right]\right)
\end{equation}

\begin{equation}
    \alpha_{ij} = \frac{\exp(e_{ij})}{\sum_{k \in \mathcal{N}_i} \exp(e_{ik})}
\end{equation}

where $\mathbf{h}_i$ represents the features of node $i$, $W \in \mathbb{R}^{d_{out} \times d_{in}}$ is a learned transformation matrix, $\mathbf{a} \in \mathbb{R}^{2d_{out}}$ is a learned attention vector, and $\mathcal{N}_i$ denotes the neighborhood of node $i$. The node representations are updated using these attention weights:

\begin{equation}
    \mathbf{h}'_i = \sigma\left(\sum_{j \in \mathcal{N}_i} \alpha_{ij} W \mathbf{h}_j\right)
\end{equation}

where $\sigma$ is a non-linear activation function (ReLU in our implementation).

To capture different types of relationships between agents, the GNN employs multi-head attention with $K=4$ attention heads:

\begin{equation}
    \mathbf{h}'_i = [\mathbf{h}^1_i; \mathbf{h}^2_i; \ldots; \mathbf{h}^K_i], \quad \text{where} \quad \mathbf{h}^k_i = \sigma\left(\sum_{j \in \mathcal{N}_i} \alpha^k_{ij} W^k \mathbf{h}_j\right)
\end{equation}

where $[\cdot; \cdot]$ denotes concatenation and each head $k$ has its own parameters $W^k$ and $\mathbf{a}^k$. This allows the model to attend to different aspects of agent relationships simultaneously.

A critical innovation in our Temporal GNN is the incorporation of temporal dependencies through a sliding window memory mechanism. The system maintains a buffer of previous graph representations over a temporal window of size $T=5$:

\begin{equation}
    \mathcal{M}_t = \{\mathbf{H}_{t-T+1}, \mathbf{H}_{t-T+2}, \ldots, \mathbf{H}_t\}
\end{equation}

where $\mathbf{H}_s$ represents the node features at time step $s$. Temporal attention is applied across this sequence using a standard transformer attention mechanism:

\begin{equation}
    \text{TemporalAttention}(\mathbf{Q}_t, \mathbf{K}_{t-T+1:t}, \mathbf{V}_{t-T+1:t}) = \text{softmax}\left(\frac{\mathbf{Q}_t\mathbf{K}_{t-T+1:t}^T}{\sqrt{d}}\right)\mathbf{V}_{t-T+1:t}
\end{equation}

This temporal attention mechanism enables the model to capture how agent behaviors evolve over time and identify patterns in their learning dynamics.

The Temporal GNN outputs parameters for a variational posterior distribution over latent states representing other agents' hidden characteristics:

\begin{align}
    \boldsymbol{\mu}_t &= W_{\mu}\mathbf{h}^{\text{final}}_i + \mathbf{b}_{\mu} \\
    \log\boldsymbol{\sigma}_t &= W_{\sigma}\mathbf{h}^{\text{final}}_i + \mathbf{b}_{\sigma}
\end{align}

where $\mathbf{h}^{\text{final}}_i$ is the final node representation for agent $i$ after all GNN layers and temporal attention. The latent variables are sampled using the reparameterization trick:

\begin{equation}
    \hat{\mathbf{z}}^{-i}_{t+1} = \boldsymbol{\mu}_t + \boldsymbol{\sigma}_t \odot \boldsymbol{\epsilon}, \quad \boldsymbol{\epsilon} \sim \mathcal{N}(\mathbf{0}, \mathbf{I})
\end{equation}

\section{Policy and Value Networks}

The policy networks map belief states and inferred latent variables to action decisions, implementing the agent's strategy for maximizing long-term rewards. For discrete action spaces, the policy network outputs categorical distributions:

\begin{align}
    \mathbf{h}^1 &= \text{ReLU}(W^1[\mathbf{b}^i_t; \hat{\mathbf{z}}^{-i}_t] + \mathbf{b}^1) \\
    \mathbf{h}^2 &= \text{ReLU}(W^2\mathbf{h}^1 + \mathbf{b}^2) \\
    \boldsymbol{\pi}^i(\cdot|\mathbf{b}^i_t, \hat{\mathbf{z}}^{-i}_t; \boldsymbol{\theta}^i) &= \text{softmax}(W^{\text{out}}\mathbf{h}^2 + \mathbf{b}^{\text{out}})
\end{align}

where $[\cdot; \cdot]$ denotes concatenation.

POLARIS employs dual Q-networks~\cite{hasselt2016deep} to mitigate overestimation bias, a critical consideration in multi-agent settings where overconfident value estimates can lead to suboptimal strategic behaviors. For discrete actions, the Q-networks take belief states, latent variables, and one-hot encoded neighbor actions:

\begin{align}
    \mathbf{x}_q &= [\mathbf{b}^i_t; \hat{\mathbf{z}}^{-i}_t; \mathbf{a}^{\text{neighbors}}_{\text{onehot}}] \\
    \mathbf{h}^1_q &= \text{ReLU}(W^1_q\mathbf{x}_q + \mathbf{b}^1_q) \\
    \mathbf{h}^2_q &= \text{ReLU}(W^2_q\mathbf{h}^1_q + \mathbf{b}^2_q) \\
    \mathbf{Q}^i &= W^{\text{out}}_q\mathbf{h}^2_q + \mathbf{b}^{\text{out}}_q
\end{align}


Target networks are maintained for stability, updated using exponential moving averages:

\begin{equation}
    \boldsymbol{\theta}_{\text{target}} \leftarrow \tau \boldsymbol{\theta} + (1-\tau)\boldsymbol{\theta}_{\text{target}}
\end{equation}

with $\tau = 0.005$ in our implementation.

\chapter{Social Learning Implementation}

This chapter presents the technical implementation details of our social learning models within the Partially Observable Active Markov Game framework. We provide a comprehensive account of the algorithmic approaches, mathematical techniques, and computational methods used to translate theoretical social learning concepts into executable simulations. The discussion begins with the rigorous discretization of Lévy processes—the mathematical foundation for modeling stochastic rewards and signals—and proceeds to cover the specific implementation challenges encountered in strategic experimentation scenarios. Another key innovation is our construction of observation-based reward functions that preserve expected reward structures when agents cannot directly observe the true environmental state, allowing them to learn optimal policies from noisy signals while maintaining the incentive properties of the original models. Throughout this chapter, we emphasize how our implementation preserves the strategic incentives inherent in the original models while adapting them to a computational framework suitable for reinforcement learning applications.

\section{Lévy Process Discretization}
\label{appendix:levy_discretization}

This appendix provides a comprehensive mathematical treatment of the discretization of Lévy processes for implementing strategic experimentation models within our Partially Observable Active Markov Game framework. We establish rigorous theoretical foundations for the numerical approximation schemes used in our implementation and analyze their convergence properties and preservation of strategic incentives.

\subsection{Mathematical Foundations of Lévy Processes}
\label{appendix:levy_foundations}

Lévy processes form a fundamental class of stochastic processes that include Brownian motion and Poisson processes as special cases. They are characterized by stationary and independent increments, serving as the natural continuous-time generalization of random walks.

\begin{definition}[Lévy Process]
    A stochastic process $X = \{X_t : t \geq 0\}$ on $\mathbb{R}^d$ with $X_0 = 0$ almost surely is a Lévy process if:
    \begin{enumerate}
        \item It has independent increments: for any $0 \leq t_1 < t_2 < \cdots < t_n < \infty$, the random variables $X_{t_2} - X_{t_1}, X_{t_3} - X_{t_2}, \ldots, X_{t_n} - X_{t_{n-1}}$ are mutually independent.
        \item It has stationary increments: for any $s < t$, the distribution of $X_t - X_s$ depends only on $t-s$.
        \item It is stochastically continuous: for any $t \geq 0$ and $\varepsilon > 0$, $\lim_{h \to 0} \mathbb{P}(|X_{t+h} - X_t| > \varepsilon) = 0$.
    \end{enumerate}
\end{definition}

The celebrated Lévy-Khintchine formula provides a complete characterization of Lévy processes through their characteristic functions:

\begin{theorem}[Lévy-Khintchine Formula \cite{sato1999levy, applebaum2009levy}]
    If $X = \{X_t : t \geq 0\}$ is a Lévy process, then its characteristic function has the form:
    \begin{equation}
        \mathbb{E}[e^{i\theta X_t}] = e^{t\psi(\theta)}
    \end{equation}
    where
    \begin{equation}
        \psi(\theta) = i\alpha\theta - \frac{1}{2}\sigma^2\theta^2 + \int_{\mathbb{R}\setminus\{0\}} (e^{i\theta x} - 1 - i\theta x\mathbf{1}_{|x|<1})\nu(dx)
    \end{equation}
    for some $\alpha \in \mathbb{R}$, $\sigma \geq 0$, and a measure $\nu$ on $\mathbb{R}\setminus\{0\}$ satisfying $\int_{\mathbb{R}\setminus\{0\}} \min(1, x^2)\nu(dx) < \infty$.
\end{theorem}

The triplet $(\alpha, \sigma^2, \nu)$ is called the Lévy-Khintchine triplet or the characteristics of the Lévy process. Here, $\alpha$ represents a drift term, $\sigma^2$ parametrizes the continuous Gaussian component, and $\nu$ is the Lévy measure characterizing the jump behavior.

\begin{theorem}[Lévy-Itô Decomposition \cite{sato1999levy, protter2005stochastic}]
    Any Lévy process $X_t$ can be decomposed as:
    \begin{equation}
        X_t = \alpha t + \sigma W_t + \int_{|x|<1} x (\tilde{Y}(t, dx) - t\nu(dx)) + \int_{|x|\geq 1} x Y(t, dx)
    \end{equation}
    where $W_t$ is a standard Brownian motion, $Y(t, A)$ counts the number of jumps of size in set $A$ occurring up to time $t$, and $\tilde{Y}(t, dx) = Y(t, dx) - t\nu(dx)$ is the compensated Poisson random measure.
\end{theorem}

The Lévy-Itô decomposition provides a pathwise representation of a Lévy process as the sum of a drift term, a Brownian motion, and potentially infinitely many jumps, both small and large.

\subsection{Time Discretization of Lévy Processes}
\label{appendix:levy_discretization_methods}

To implement continuous-time Lévy processes within discrete computational frameworks, we must employ appropriate numerical approximation schemes. For our strategic experimentation model, we adopt the Euler-Maruyama scheme, extended to accommodate the jump components of general Lévy processes.

\begin{definition}[Euler-Maruyama Scheme for Lévy Processes]
    Given a Lévy process $X_t$ with characteristics $(\alpha, \sigma^2, \nu)$ and a discretization time step $\Delta t$, the Euler-Maruyama approximation constructs a discrete-time process $\{X_{t_n}\}_{n=0}^N$ where $t_n = n\Delta t$ through the recursive relation:
    \begin{equation}
        X_{t_{n+1}} = X_{t_n} + \alpha \Delta t + \sigma\sqrt{\Delta t}Z_n + \Delta J_n
    \end{equation}
    where $Z_n \sim \mathcal{N}(0,1)$ are independent standard normal random variables and $\Delta J_n$ represents the jump increment over $[t_n, t_{n+1}]$.
\end{definition}
\iffalse
    For practical implementation, the jump component $\Delta J_n$ requires careful consideration, especially when the Lévy measure potentially assigns infinite mass to small jumps.

    \begin{proposition}[Approximation of Jump Component \cite{platen2010numerical, applebaum2009levy}]
        The jump component $\Delta J_n$ can be approximated through one of the following approaches:
        \begin{enumerate}
            \item For finite-activity processes (where $\nu(\mathbb{R}) < \infty$):
                  \begin{equation}
                      \Delta J_n = \sum_{i=1}^{K_n} J_i
                  \end{equation}
                  where $K_n \sim \text{Poisson}(\nu(\mathbb{R})\Delta t)$ and $J_i$ are i.i.d. random variables with distribution $\frac{\nu(dx)}{\nu(\mathbb{R})}$.

            \item For infinite-activity processes, a truncation approach:
                  \begin{equation}
                      \Delta J_n = \sum_{i=1}^{K^{\varepsilon}_n} J^{\varepsilon}_i + c_{\varepsilon}\Delta t
                  \end{equation}
                  where $K^{\varepsilon}_n \sim \text{Poisson}(\nu(\{x: |x| > \varepsilon\})\Delta t)$, $J^{\varepsilon}_i$ are i.i.d. with distribution $\frac{\nu(dx)\mathbf{1}_{|x|>\varepsilon}}{\nu(\{x: |x| > \varepsilon\})}$, and $c_{\varepsilon} = \int_{|x|\leq\varepsilon} x\nu(dx)$.
        \end{enumerate}
    \end{proposition}
\fi
\begin{theorem}[Convergence of Euler-Maruyama Scheme \citep{higham2005numerical, platen2010numerical}]
    \label{thm:euler_maruyama_convergence}
    Let $X_t$ be a Lévy process and $\hat{X}_t$ be its Euler-Maruyama approximation with time step $\Delta t$. Then for any fixed $T > 0$:
    \begin{enumerate}
        \item (Weak Convergence) For any smooth function $f$ with polynomial growth:
              \begin{equation}
                  |\mathbb{E}[f(X_T)] - \mathbb{E}[f(\hat{X}_T)]| \leq C\Delta t
              \end{equation}

        \item (Strong Convergence) If $\int_{|x|>1} |x|^2 \nu(dx) < \infty$, then:
              \begin{equation}
                  \mathbb{E}[\sup_{0\leq t\leq T} |X_t - \hat{X}_t|^2] \leq C\Delta t
              \end{equation}
    \end{enumerate}
    where $C$ is a constant depending on $T$ and the characteristics of the Lévy process.

\end{theorem}

The weak and strong convergence properties ensure that our numerical scheme accurately approximates both the distributional properties and pathwise behavior of the continuous-time process as the time step decreases.

\subsection{Implementing Strategic Experimentation Models}
\label{appendix:strategic_experimentation_implementation}

In our implementation of the strategic experimentation model from \citet{keller2020undiscounted}, we must discretize both the background signal process $B_t$ and the individual payoff processes $X^i_t$, while preserving the strategic incentives that drive experimentation decisions.

\subsubsection{Discretization of Diffusion-Poisson Processes}
\label{appendix:discretization_diffusion_poisson}

In the original model, both $B_t$ and $X^i_t$ follow Lévy processes that combine continuous diffusion and discrete jumps:

\begin{align}
    dB_t   & = \beta_s dt + \sigma_B dW^B_t + dY^B_t \\
    dX^i_t & = \alpha_s dt + \sigma dW^i_t + dY^i_t
\end{align}

where $W^B_t$ and $W^i_t$ are standard Brownian motions, and $Y^B_t$ and $Y^i_t$ are compound Poisson processes. Using the Euler-Maruyama scheme \citep{platen2010numerical}, we discretize these continuous-time stochastic differential equations:

\begin{align}
    B_{t+\Delta t} - B_t     & = \beta_s \Delta t + \sigma_B (W^B_{t+\Delta t} - W^B_t) + (Y^B_{t+\Delta t} - Y^B_t) \\
    X^i_{t+\Delta t} - X^i_t & = \alpha_s \Delta t + \sigma (W^i_{t+\Delta t} - W^i_t) + (Y^i_{t+\Delta t} - Y^i_t)
\end{align}

For implementation, we denote these increments as:

\begin{align}
    B_{t-1:t}   & = \beta_s \Delta t + \sigma_B (W^B_t - W^B_{t-1}) + \Delta Y^B_t \\
    X^i_{t-1:t} & = \alpha_s \Delta t + \sigma (W^i_t - W^i_{t-1}) + \Delta Y^i_t
\end{align}

where $(W^B_t - W^B_{t-1}) \sim \mathcal{N}(0, \Delta t)$, $(W^i_t - W^i_{t-1}) \sim \mathcal{N}(0, \Delta t)$, and $\Delta Y^B_t$, $\Delta Y^i_t$ are the increments of the compound Poisson processes over the interval $[t-1, t]$.

\subsubsection{Reward Function Equivalence}
\label{appendix:reward_equivalence}

A critical challenge in our implementation is reconciling the time-dependent nature of continuous-time rewards with the time-independent reward structure required by the POAMG framework. In the original continuous-time model, agents' instantaneous rewards are:

\begin{equation}
    dR^i_t = (1-a^i_t)r_\textit{safe} dt + a^i_t dX^i_t
\end{equation}

where $a^i_t \in [0,1]$ is the allocation to the risky arm and $r_\textit{safe}$ is the safe arm's deterministic flow payoff and the Levy process increment is a function of the state $s$. To translate this structure into the POAMG framework, we need a reward function $R^i(s, a^i)$ that depends only on the state and action, not explicitly on time. We achieve this through a normalization approach:

\begin{equation}
    R^i(s, a^i) = (1-a^i)r_\textit{safe} + a^i \frac{X^i_{t-1:t}}{\Delta t}
\end{equation}

This transformation preserves the incentive structure of the original model while eliminating explicit time dependence. The following proposition establishes this equivalence formally:

\begin{proposition}[Reward Equivalence]
    The expected value of the discrete-time reward function $R^i(s, a^i)$ exactly equals the expected instantaneous flow payoff in the continuous-time model. Specifically:
    \begin{equation}
        \mathbb{E}[R^i(s, a^i)] = (1-a^i)r_\textit{safe} + a^i(\alpha_s + \lambda_s h_s)
    \end{equation}
    where $\alpha_s$ is the drift of the Levy process, $\lambda_s$ is the jump intensity and $h_s$ is the mean jump size of the compound Poisson process component of $X^i_t$ in state $s$.
\end{proposition}

\begin{proof}
    The expected value of the discrete-time reward function is:
    \begin{align}
        \mathbb{E}[R^i(s, a^i)] & = (1-a^i)r_\textit{safe} + a^i\mathbb{E}\left[\frac{X^i_{t-1:t}}{\Delta t}\right]                                       \\
                                & = (1-a^i)r_\textit{safe} + a^i\frac{\mathbb{E}[\alpha_s \Delta t + \sigma(W^i_t - W^i_{t-1}) + \Delta Y^i_t]}{\Delta t} \\
                                & = (1-a^i)r_\textit{safe} + a^i\left(\alpha_s + \frac{\mathbb{E}[\Delta Y^i_t]}{\Delta t}\right)
    \end{align}

    Since $\mathbb{E}[W^i_t - W^i_{t-1}] = 0$ and $\mathbb{E}[\Delta Y^i_t] = \lambda_s h_s\Delta t$, where $\lambda_s$ is the jump intensity and $h_s$ is the mean jump size, we have:

    \begin{align}
        \mathbb{E}[R^i(s, a^i)] & = (1-a^i)r_\textit{safe} + a^i\left(\alpha_s + \frac{\lambda_s h_s\Delta t}{\Delta t}\right) \\
                                & = (1-a^i)r_\textit{safe} + a^i(\alpha_s + \lambda_s h_s)
    \end{align}

    This exactly matches the expected instantaneous flow payoff in the continuous-time model.
\end{proof}


\iffalse
    \subsubsection{Preservation of Strategic Incentives}
    \label{appendix:strategic_incentives}

    Beyond matching expected rewards, we must ensure that our discretization preserves the strategic incentives that drive experimentation decisions. The key strategic consideration in the model is the trade-off between exploitation (choosing the safe arm) and exploration (choosing the risky arm to learn about its quality).

    \begin{proposition}[Preservation of Strategic Incentives]
        The discrete-time reward function $R^i(s, a^i)$ preserves the strategic incentives of the continuous-time model in the sense that the optimal policy in the discrete-time model converges to the optimal policy in the continuous-time model as $\Delta t \to 0$.

    \end{proposition}

    \begin{proof}
        In the strategic experimentation model, the incentive to experiment at a given belief $b$ about the state is characterized by:
        \begin{equation}
            I(b) = \frac{f(b) - s}{s - m(b)}
        \end{equation}
        when $m(b) < s$, and $I(b) = \infty$ otherwise, where $f(b)$ is the expected flow payoff of the risky arm and $m(b)$ is the expected flow payoff when the risky arm is bad. In our discrete-time approximation, the corresponding incentive is:
        \begin{equation}
            \hat{I}(b) = \frac{\hat{f}(b) - s}{s - \hat{m}(b)}
        \end{equation}
        where $\hat{f}(b)$ and $\hat{m}(b)$ are the discrete-time analogs.

        Let us elaborate on why $\hat{f}(b) \to f(b)$ and $\hat{m}(b) \to m(b)$ as $\Delta t \to 0$. In the continuous-time model, the expected flow payoff of the risky arm at belief $b$ is:
        \begin{equation}
            f(b) = b f_1 + (1-b) f_0
        \end{equation}
        where $f_1 = \alpha_1 + \lambda_1\mu_1$ is the expected flow in the good state and $f_0 = \alpha_0 + \lambda_0\mu_0$ is the expected flow in the bad state, with $m(b) = f_0$. In our discrete-time approximation, we have:
        \begin{equation}
            \hat{f}(b) = b \hat{f}_1 + (1-b) \hat{f}_0
        \end{equation}
        where $\hat{f}_1$ and $\hat{f}_0$ are the expected rewards in the good and bad states, respectively.
        From our reward equivalence result, we know that:
        \begin{align}
            \hat{f}_1 & = \mathbb{E}\left[\frac{X^i_{t-1:t}}{\Delta t} \mid \text{good state}\right] = \alpha_1 + \frac{\mathbb{E}[\sigma(W^i_t - W^i_{t-1}) + \Delta N^i_t \mid \text{good state}]}{\Delta t} \\
                      & = \alpha_1 + \frac{\lambda_1\mu_1\Delta t}{\Delta t} + \frac{\mathbb{E}[\sigma(W^i_t - W^i_{t-1})]}{\Delta t}
        \end{align}

        The Brownian motion term has $\mathbb{E}[\sigma(W^i_t - W^i_{t-1})] = 0$ and $\text{Var}[\sigma(W^i_t - W^i_{t-1})] = \sigma^2 \Delta t$. By the martingale property of Brownian motion, this term contributes zero to the expected value but introduces variance of order $O(\Delta t)$. Thus:
        \begin{equation}
            \hat{f}_1 = \alpha_1 + \lambda_1\mu_1 + O(\Delta t) \to f_1 \text{ as } \Delta t \to 0
        \end{equation}

        Similarly, $\hat{f}_0 = \alpha_0 + \lambda_0\mu_0 + O(\Delta t) \to f_0 \text{ as } \Delta t \to 0$, and consequently $\hat{m}(b) = \hat{f}_0 \to m(b) = f_0$.
        Therefore, we can establish that:
        \begin{equation}
            \hat{I}(b) = \frac{\hat{f}(b) - s}{s - \hat{m}(b)} = \frac{b \hat{f}_1 + (1-b) \hat{f}_0 - s}{s - \hat{f}_0} = \frac{b(\hat{f}_1 - \hat{f}_0) + \hat{f}_0 - s}{s - \hat{f}_0}
        \end{equation}

        As $\Delta t \to 0$, we have $\hat{f}_1 \to f_1$, $\hat{f}_0 \to f_0$, and consequently:
        \begin{equation}
            \hat{I}(b) \to \frac{b(f_1 - f_0) + f_0 - s}{s - f_0} = \frac{b f_1 + (1-b) f_0 - s}{s - f_0} = \frac{f(b) - s}{s - m(b)} = I(b)
        \end{equation}

        The convergence of incentives directly implies convergence of optimal policies. In the continuous-time model, the symmetric Markov perfect equilibrium characterized in \citet{keller2020undiscounted} has the form:
        \begin{equation}
            \kappa^*(b) =
            \begin{cases}
                0                    & \text{if } I(b) \leq k_0,            \\
                \frac{I(b)-k_0}{N-1} & \text{if } k_0 < I(b) < k_0 + N - 1, \\
                1                    & \text{if } I(b) \geq k_0 + N - 1.
            \end{cases}
        \end{equation}
        where $k_0$ is a threshold determined by the model parameters and $N$ is the number of players. The discrete-time counterpart $\hat{\kappa}^*(b)$ takes the same form but with $\hat{I}(b)$ in place of $I(b)$.

        Therefore, as $\Delta t \to 0$, the strategic incentives and optimal policies in our discrete-time approximation converge to those of the original continuous-time model.
    \end{proof}

\fi
\subsection{Relation to Policy-Invariant Reward Transformations}
\label{appendix:policy_invariance}

Our approach to time-independent reward formulation can be viewed through the lens of policy-invariant reward transformations, as developed in the reinforcement learning literature by \citet{ng1999policy}.

\begin{definition}[Potential-Based Reward Shaping]
    A reward transformation $\tilde{R}(s, a, s') = R(s, a, s') + F(s, s')$ is potential-based if there exists a potential function $\Phi: S \to \mathbb{R}$ such that $F(s, s') = \gamma\Phi(s') - \Phi(s)$ for all $s, s' \in S$, where $\gamma$ is a discount factor.
\end{definition}

\begin{theorem}[Ng-Harada-Russell \citep{ng1999policy}]
    If a reward transformation is potential-based, then the optimal policy under the transformed reward function is also optimal under the original reward function, and vice versa.
\end{theorem}

While our transformation does not directly fit the potential-based formulation (since we're normalizing by $\Delta t$ rather than adding a potential difference), it shares the crucial property of preserving optimal policies. The following result establishes this connection:

\begin{proposition}[Connection to Policy-Invariant Transformations]
    The transformation from time-dependent rewards in the continuous-time model to time-independent rewards in the POAMG framework preserves policy optimality in the limit as $\Delta t \to 0$.
\end{proposition}

\begin{proof}
    The original cumulative reward over a time interval $[0, T]$ is:
    \begin{equation}
        \int_0^T [(1-a^i_t)s + a^i_t\alpha_s] dt + \int_0^T a^i_t \sigma dW^i_t + \int_0^T a^i_t dC^i_t
    \end{equation}

    The discretized cumulative reward is:
    \begin{equation}
        \sum_{k=0}^{\eta-1} \Delta t \left[(1-a^i_{t_k})s + a^i_{t_k}\frac{X^i_{t_k:t_{k+1}}}{\Delta t}\right] = \sum_{k=0}^{\eta-1} [(1-a^i_{t_k})s\Delta t + a^i_{t_k}X^i_{t_k:t_{k+1}}]
    \end{equation}
    where $\eta = T/\Delta t$ and $t_k = k\Delta t$.

    As $\Delta t \to 0$ and $\eta \to \infty$ with $\eta t = T$ fixed, the discretized sum converges to the continuous-time integral. Since both formulations yield the same expected cumulative reward in the limit, they induce the same optimal policies.
\end{proof}


\section{Observed Reward Function}
\label{appendix:observed_reward_derivation} In our reinforcement learning
implementation, agents need to learn from observed rewards that reflect the optimality
of their actions with respect to the true state of the world. However, since agents
do not directly observe the true state $\omega$, we need to construct a reward
function based on their observations $o$ that preserves the expected reward
structure. This appendix provides a detailed derivation of the observation-based
reward function.The true reward function is given by $u(s, a): S \times A \rightarrow
    \mathbb{R}$, where $a$ is an action and $s$ is the state. However, agents only
observe signals $o \sim \mu^{s}$ drawn from a distribution determined by the
state. We need to construct an observation-based reward function $v(o, a): \Omega
    _{i} \times A \rightarrow \mathbb{R}$ that satisfies:
\begin{equation}
    \mathbb{E}_{o \sim \mu^{s}}[v(o, a)] = u(s, a), \quad \forall s \in S, \forall
    a \in A.
\end{equation}

In other words, the expected value of the observation-based reward $v(o, a)$
over all possible observations $o$ given the state $s$ should equal the true
reward $u(s, a)$ for any state-action pair. Let's define $\bm{\mu}$ as a $m \times
    m$ matrix representing the observation distributions, where $m = |S|$ and assume
the aforementioned state and observation spaces. Each entry $\mu^{s}(o)$ denotes
the probability of observing observation $o$ in state $s$:
\begin{equation}
    \bm{\mu}=
    \begin{bmatrix}
        \mu^{s^1}(s^{1}) & \mu^{s^1}(s^{2}) & \cdots & \mu^{s^1}(s^{m}) \\
        \mu^{s^2}(s^{1}) & \mu^{s^2}(s^{2}) & \cdots & \mu^{s^2}(s^{m}) \\
        \vdots           & \vdots           & \ddots & \vdots           \\
        \mu^{s^m}(s^{1}) & \mu^{s^m}(s^{2}) & \cdots & \mu^{s^m}(s^{m})
    \end{bmatrix}
\end{equation}

We assume that observation distributions are linearly independent across states,
which implies that $\bm{\mu}$ is invertible. This assumption is satisfied when $\mu
    ^{s}(s) > \mu^{s}(s')$ for all $s \neq s' \in S$, meaning that each state is
more likely to generate its corresponding observation than any other observation.
For each action $a \in A$, we define a utility vector
$\bm{u}_{a} \in \mathbb{R}^{k}$ as:
\begin{equation}
    \bm{u}_{a} =
    \begin{bmatrix}
        u(s^{1}, a) \\
        u(s^{2}, a) \\
        \vdots      \\
        u(s^{m}, a)
    \end{bmatrix}
\end{equation}

Similarly, we define a vector of observation-based rewards $\bm{v}_{a} \in \mathbb{R}
    ^{k}$ for action $a$:
\begin{equation}
    \bm{v}_{a} =
    \begin{bmatrix}
        v(s^{1}, a) \\
        v(s^{2}, a) \\
        \vdots      \\
        v(s^{m}, a)
    \end{bmatrix}
\end{equation}

The condition that expected observation-based rewards match true rewards can be written
as a linear system:
\begin{equation}
    \bm{\mu}^{T} \bm{v}_{a} = \bm{u}_{a}
\end{equation}

where $\bm{\mu}^{T}$ is the transpose of $\bm{\mu}$. The solution to this system
is:
\begin{equation}
    \bm{v}_{a} = (\bm{\mu}^{T})^{-1}\bm{u}_{a} = (\bm{\mu}^{-1})^{T} \bm{u}_{a}
\end{equation}

This gives us the vector of observation-based rewards that satisfy our expected
reward condition. For each observation $o_{j}$ and action $a$, the observation-based
reward is:
\begin{equation}
    v(o_{j}, a) = \sum_{l=1}^{m} \left( \bm{\mu}^{-1}\right)_{jl}u(s_{l}, a)
\end{equation}

where $\left( \bm{\mu}^{-1}\right)_{jl}$ denotes the $(j, l)$-th element of
$\bm{\mu}^{-1}$.
\paragraph{Binary Case Example}
For the special case with two states of the world ($m = 2$) and symmetric
observations with accuracy $q > 0.5$, the observation distribution matrix is:
\begin{equation}
    \bm{\mu}=
    \begin{bmatrix}
        q   & 1-q \\
        1-q & q
    \end{bmatrix}
\end{equation}

This means that in state $s^{1}$, the probability of observing $s^{1}$ is $q$
and the probability of observing $s^{2}$ is $1-q$, and vice versa for state $s^{2}$.
The inverse of this matrix is:
\begin{equation}
    \bm{\mu}^{-1}= \frac{1}{2q-1}
    \begin{bmatrix}
        q      & -(1-q) \\
        -(1-q) & q
    \end{bmatrix}
\end{equation}

In our implementation, we set the true reward as the indicator function for taking
the correct action, which for the binary case becomes, $u(s^{1}, a^{1}) = u(s^{2}
    , a^{2}) = 1$; $u(s^{2}, a^{1}) = u(s^{1}, a^{2}) = 0$. For action $a^{1}$, the
utility vector is $\bm{u}_{a^1}= [1, 0]^{T}$, and the observation-based reward vector
is:
\begin{equation}
    \bm{v}_{a^1}= \bm{\mu}^{-1}\bm{u}_{a^1}= \frac{1}{2q-1}
    \begin{bmatrix}
        q \\
        -(1-q)
    \end{bmatrix}
\end{equation}

This means that when an agent takes action $a^{1}$ and observes signal $s^{1}$,
the reward is $v(s^{1}, a^{1}) = \frac{q}{2q-1}$, and when the agent observes signal
$s^{2}$, the reward is $v(s^{2}, a^{1}) = -\frac{1-q}{2q-1}$. Similarly, for
action $a^{2}$, the utility vector is $\bm{u}_{a^2}= [0, 1]^{T}$, and the
observation-based reward vector is:
\begin{equation}
    \bm{v}_{a^2}= \bm{\mu}^{-1}\bm{u}_{a^2}= \frac{1}{2q-1}
    \begin{bmatrix}
        -(1-q) \\
        q
    \end{bmatrix}
\end{equation}

Combining these results, we can express the observation-based reward function for
any action-observation pair $(a, o)$ as:
\begin{equation}
    v(a, o) = \frac{q \cdot \mathbf{1}_{\{a = \varphi(o)\}}- (1-q) \cdot
        \mathbf{1}_{\{a \neq \varphi(o)\}}}{2q-1}
\end{equation}

where $\varphi$ maps observations to their corresponding actions. The derived
observation-based reward function has an intuitive interpretation. When an agent
takes an action matching its observation ($a = o$), it receives a positive
reward of $\frac{q}{2q-1}$, whereas when she takes an action not matching its observation
($a \neq o$), it receives a negative reward of $-\frac{1-q}{2q-1}$. As $q$ approaches
0.5 (signals become uninformative), the rewards grow in magnitude, reflecting the
increased uncertainty. Conversely, as $q$ approaches 1 (signals become perfectly
informative), the reward for matching observation and action approaches 1, and the
penalty for mismatching approaches 0, meaning that the observed reward function
converges to the true reward function. This reward structure incentivizes agents
to match their actions to their observations when signals are reliable, but also
allows them to learn to override their immediate observations when the actions
of other agents provide stronger evidence about the true state through beliefs.

