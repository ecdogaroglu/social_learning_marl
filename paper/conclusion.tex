This thesis bridges economic social learning theory and multi-agent reinforcement learning by introducing Partially Observable Active Markov Games (POAMGs) and the POLARIS algorithm. Our framework addresses fundamental challenges in modeling strategic adaptation under partial observability, where agents must simultaneously learn about their environment and strategically influence others' learning processes.

Our theoretical contributions establish POAMGs as a rigorous mathematical framework extending Active Markov Games to partially observable settings. The convergence analysis and policy gradient theorems provide solid foundations for the POLARIS algorithm, which successfully implements this framework through an integrated architecture combining belief processing, inference learning, and reinforcement learning modules.

Experimental validation across strategic experimentation and learning without experimentation scenarios demonstrates our framework's effectiveness. POLARIS agents converge to theoretically predicted equilibrium strategies, validating key insights about free-riding behavior and optimal experimentation intensity. Our results confirm the existence of social learning barriers while revealing coordination benefits in larger networks. Crucially, we uncover a dynamic division of labor where agents assume roles as information generators or exploiters based on signal quality rather than network position.

Future research could extend the framework to incorporate heterogeneous agent types, dynamic network structures, or mechanism design principles. Applications to contemporary challenges such as misinformation spread or algorithmic recommendation systems could demonstrate broader relevance.

This work demonstrates that integrating economic theory and computational methods yields insights neither approach could achieve alone. The POAMG framework and POLARIS algorithm provide both theoretical understanding and practical tools for analyzing complex social learning processes, establishing a foundation for future interdisciplinary research. As multi-agent systems become increasingly prevalent, such integrated approaches will be essential for understanding and shaping collective intelligence emerging from strategic interactions among learning agents.