\subsection*{Social Learning in Networks}

We adapt the model in Brandl (2024). Specifically, we let $N = \{1, \ldots, n\}$ be the set of agents, and $T = \{1, 2, \ldots, t \}$ be the finite set of periods. In each period, each agent chooses an action $a$ from the same finite set of actions $A$. The finite set of states of the world is $\Omega$, where the true state $\omega \in \Omega$ is a random variable with full support distribution over $\Omega$. Agents share the same utility function $u : A \times \Omega \to \mathbb{R}$ that depends on their own action and the state of the world, where $a_\theta$ denotes the unique optimal (correct) action in that state. It's also assumed that $a_\theta \neq a_{\theta'}$ for any two distinct states $\theta, \theta' \in \Omega$. In each period $t \in T$, each agent $i$ privately observes a signal $s_t^i$ from a set of signals $S$. Conditional on each state $\theta$, $s_t^i$ has distribution $\mu_\theta \in \Delta(S)$, and signals are independent across agents and periods. Observing the signal realization $s \in S$ changes the log-likelihood ratio of the observing agent between the states $\theta$ and $\theta'$ by,
\[
\ell_{\theta, \theta'}(s) = \log \frac{d\mu_\theta(s)}{d\mu_{\theta'}(s)}.
\]
Each agent $i$ also observes the actions of her neighbors $N^i \subset N$, thus the information available to Agent $i$ in period $t$ before choosing an action is the collection of information sets $\mathcal{I}^i_{\leq t} = S^t \times A^{|N^i| \times (t-1)}$. It's said that Agent $i$ makes a mistake in period $t$ if $a_t^i \neq a_\omega$, and that $i$ learns at rate $r$ if,
\[
r = \liminf_{t \to \infty} -\frac{1}{t} \log P(a_t^i \neq a_\omega).
\]
Theorem 1 states that for any number of agents and any strategy profile, there exists an agent $i$ such that,
\[r^i \leq r_{bdd} = \min_{\theta \neq \theta'}\{E_\theta[\ell_{\theta,\theta'}] + E_{\theta'}[\ell_{\theta',\theta}]\}.\]


\subsection*{Partially Observable Markov Games}
Following the MARL literature, we model the social learning framework as a partially observable Markov game defined by the tuple,
\[(N, \Psi, A, \{O_i\}_{i \in N}, \Gamma, R, \gamma)\]
where,
\begin{itemize}
\item $N$ is the set of agents,
\item $\Psi $ is the Markov state space, with individual element $\psi$,
\item $A$ is the shared action space for all agents, with individual element $a$,
\item $O_i$ is the observation space for agent $i$, with individual element $o_i$,
\item $\Gamma : \Psi \times A^n \times \Psi \to [0,1]$ is the transition function,
\item $R: A \times \Psi \to \mathbb{R}$ is the shared reward function for all agents,
\item $\gamma \in [0,1)$ is the discount factor.
\end{itemize}

\subsubsection*{Markov States, Signals \& Actions}
For simplicity, we let $\Omega = \{\theta_1, \ldots, \theta_k\}$ be the set of possible states of the world; $S = \{s_k : \theta_k \in \Omega\}$ be the signal space, with $s_k$ being the signal for state $\theta_k$ and $A = \{a_k : \theta_k \in \Omega\}$ be the action space, with $a_k$ being the optimal action for state $\theta_k$. We can then define the Markov state space as 
\begin{equation*}
    \Psi = \{\omega\} \times S \times A^n
\end{equation*}
which accounts for the true state of the world, current period signal and current period actions. We leave the tracking of signal and action histories to the Recurrent Neural Networks (RNNs), thereby avoiding exponentially increasing Markov state space formulations.

\subsubsection*{Observations}
The observation for each agent $i$ at time $t$ is,
\begin{equation*}
o^i_t = (s^i_t, \bm{a_t}) \in O_i,
\end{equation*}
where $O_i = S \times A^n$ and $\bm{a_t} = \{a_t^1, \ldots, a_t^n\}$ is the action profile at period $t$.

\subsubsection*{Transitions}
The stochastic components of the Markov state transitions are the signal realizations and the actions taken by the agents. Formally, for state 
$\psi_t = (\omega, (s^i_t)_{i \in N}, \bm{a_t})$ and next state ${}\psi_{t+1} = (\omega, (s^i_{t+1})_{i \in N}, \bm{a_{t+1}} )$, the transition function can be written as,
\begin{equation*}
    \Gamma(\psi_{t+1}|\psi_t) = \prod_{i \in N} \mu_{\omega}(s^i_{t+1}) \prod_{i \in N} \sigma_t^i(\mathcal{I}^i_{\leq t})(a^i_{t+1})
\end{equation*}
where $\sigma_t^i$ is the \textit{policy} or \textit{strategy} of agent $i$ at period $t$. There are two important points to consider here. First, in this formulation, we condition the policies on action and signal histories that are not part of the Markov state space. The fact that we can still use this information to compute the policies stems from the technology of RNNs that allows for a memory buffer from past observations, without explicitly storing the observations themselves. Second, diverging from the original paper, we allow for policies to be time-dependent, which corresponds to the learning process of the RL agents.
\subsubsection*{Rewards}
The reward functions are identical across agents with $R_i(a, \psi)=R(a,\psi), \: \forall i \in N$. The true reward is the flow payoff $u(a^i_t, \omega)$, where $u: A \times \Omega \rightarrow \mathbb{R}$ is known but its realizations are not observed. Lacking this observation, we follow Huang et al. (2024) and construct an observed payoff function $v(a, s)$ with the property that, 
\[\mathbb{E}_{s \sim \mu^{\theta}}[v(a, s)] = u(a, \theta), \quad \forall \theta \in \Omega.\] Setting $u(a, \omega) = \mathbf{1}_{\{a=a^{\omega}\}}$ the problem of constructing such a function amounts to solving a system of $|\Omega|$ linear equations for each action.
More precisely, let $\bm{\mu}$ be an $k \times k$ matrix representing the signal distributions, where the entry $\mu^{\theta}(s)$ denotes the probability of observing signal $s$ in state $\theta$. We assume $\mu^{\theta}(s_{\theta}) > \mu^{\theta}(s)$, $\forall s \in S$, $\theta \in \Omega$, which implies that signal distributions are linearly independent across states and $\bm{\mu}$ is invertible. Similarly, for each action $a \in A$, define the \textit{utility vector} $\bm{u}_a \in \mathbb{R}^k$ as,
\begin{equation*}
    \bm{u}_a = \begin{bmatrix} u(a, \theta_1) \\ u(a, \theta_2) \\ \vdots \\ u(a, \theta_k) \end{bmatrix}.
\end{equation*}
The vector of observable reward functions $\bm{v}_a \in \mathbb{R}^n$ for action $a$ is the unique solution to the linear system as in,
\begin{align*}
    \bm{\mu} \bm{v}_a &= \bm{u}_a, \\
    \bm{v}_a &= \bm{\mu}^{-1} \bm{u}_a.
\end{align*}
Hence, the reward $v(a, s_j)$ for choosing action $a$ when observing signal $s_j$ is,
\begin{equation*}
    v(a, s_j) = \sum_{l=1}^k \left( \bm{\mu}^{-1} \right)_{jl} u(a, \theta_l),
\end{equation*}
where $\left( \bm{\mu}^{-1} \right)_{jl}$ denotes the $(j, l)$-th element of $\bm{\mu}^{-1}$.

\paragraph*{Binary Case with $|\Omega|=2$.}

For the case with two states of the world and symmetric signals with accuracy $q > 0.5$ we have,
\begin{equation*}
    \bm{\mu} = \begin{bmatrix} q & 1 - q \\ 1 - q & q \end{bmatrix}, \quad \bm{\mu}^{-1} = \frac{1}{2q - 1} \begin{bmatrix} q & -(1 - q) \\ -(1 - q) & q \end{bmatrix}
\end{equation*}
Hence, for action $a$ and flow utility $u(a, \omega) = \mathbf{1}_{\{a = \omega\}}$, the observed reward function becomes,
\begin{equation*}
    R(a, \psi) = v(a, s) = \frac{q \cdot \mathbf{1}_{\{a = s\}} - (1 - q) \cdot \mathbf{1}_{\{a \neq s\}}}{2q - 1}.
\end{equation*}

\subsection*{Recurrent Neural Networks}
Before proceeding with the RNN architecture and learning framework, we define the key activation functions and operations:

The hyperbolic tangent function $\tanh: \mathbb{R} \to (-1,1)$ is defined as,
\[
    \tanh(x) = \frac{e^x - e^{-x}}{e^x + e^{-x}}.
\]

The sigmoid function $\text{sigmoid} : \mathbb{R} \to (0,1)$ is defined as,
\[
    \sigma(x) = \frac{1}{1 + e^{-x}}.
\]

The softmax function $\text{softmax}: \mathbb{R}^k \to \Delta^{k-1}$ maps a vector to a probability distribution,
\[
    \text{softmax}(x)_i = \frac{e^{x_i}}{\sum_{j=1}^k e^{x_j}}.
\]

The Hadamard (element-wise) product $\odot: \mathbb{R}^n \times \mathbb{R}^n \to \mathbb{R}^n$ is defined as,
\[
    (x \odot y)_i = x_i y_i.
\]
Following our partial observability setting, each agent $i$ observes both their private signal $s^i_t$ and the action profile $\mathbf{a}_t = (a^1_t, \ldots, a^n_t)$ at time $t$. The collection of these observations up to time $t$ forms the information set $\mathcal{I}^i_{\leq t} = S^t \times A^{|N_i|×(t-1)}$. Rather than including this growing information set in our Markov state space, we employ Recurrent Neural Networks to maintain a compact representation while preserving the essential historical information.

\subsubsection*{RNN Architecture for Information Processing}
For each agent $i$, we implement an RNN that processes the sequence of observations $o_t^i = (s_t^i, \mathbf{a}_t) \in \mathcal{O}_i$ to produce a hidden state $h_t^i \in \mathbb{R}^d$ that encodes the information set $\mathcal{I}_{\leq t}^i$. The RNN updates its hidden state through a recursive function,
\begin{equation*}
h_t^i = f_{\eta_i}(h_{t-1}^i, o_t^i),
\end{equation*}
where $f_{\eta_i}$ is parameterized by $\eta_i$ and implemented using a Gated Recurrent Unit (GRU) architecture, following \cite{cho2014learning}. The GRU utilizes two primary gates - an update gate and a reset gate - to selectively incorporate new information while preserving relevant historical context:
\begin{align*}
z_t^i &= \text{sigmoid}(W_z^i[h_{t-1}^i, s_t^i, \mathbf{a}_t] + b_z^i) & \text{(update gate)} \\
\rho_t^i &= \text{sigmoid}(W_\rho^i[h_{t-1}^i, s_t^i, \mathbf{a}_t] + b_\rho^i) & \text{(reset gate)} \\
\tilde{h}_t^i &= \tanh(W_h^i[\rho_t^i \odot h_{t-1}^i, s_t^i, \mathbf{a}_t] + b_h^i) & \text{(candidate state)} \\
h_t^i &= (1-z_t^i) \odot h_{t-1}^i + z_t^i \odot \tilde{h}_t^i & \text{(final update)}
\end{align*}
Here, $W_z^i, W_\rho^i, W_h^i$ are weight matrices and $b_z^i, b_\rho^i, b_h^i$ are bias vectors specific to agent $i$.

The update gate $z_t^i$ determines how much of the previous hidden state should be retained versus updated with new information. When $z_t^i$ is close to 1, the network prioritizes the new candidate state $\tilde{h}_t^i$; when close to 0, it maintains more of the previous state $h_{t-1}^i$. This mechanism is particularly important in our social learning context, as it allows agents to appropriately weight new signals against accumulated historical evidence.

The reset gate $\rho_t^i$ controls how much of the previous hidden state should influence the computation of the new candidate state $\tilde{h}_t^i$. When $\rho_t^i$ is close to 0, it effectively "forgets" the previous state, allowing the network to drop irrelevant historical information. This is crucial when the agent needs to adapt to significant changes in the environment or when previous beliefs need to be revised based on new, strong evidence.

The candidate state $\tilde{h}_t^i$ is computed using the reset gate-modulated previous state along with current observations. The hyperbolic tangent activation ensures the candidate state values remain bounded between -1 and 1, promoting stable learning dynamics. The strategy or policy $\sigma_t^i$ of agent $i$ at time $t$ is then derived from this hidden state through a policy network,

\begin{equation*}
\sigma_t^i(\mathcal{I}_{\leq t}^i) = \pi_{\phi_i}(h_t^i),
\end{equation*}
where $\pi^i_{\phi_i}$ is a neural network with parameters $\phi_i$ that maps the hidden state to a distribution over actions,

\begin{equation*}
    \pi^i_{\phi_i}(h^i_t) = \text{softmax}(W^i_{\pi}h^i_t + b^i_{\pi}).
\end{equation*}
This architecture effectively augments our original Markov state space $\Psi = \{\omega\} \times S \times A^n$ with the RNN hidden states,

\begin{equation*}
    \tilde{\Psi} = \{\omega\} \times S \times A^n \times \mathcal{H}^n,
\end{equation*}
where $\mathcal{H} \subset \mathbb{R}^d$ is the space of hidden states. The transition function $\Gamma$ is accordingly modified to include the hidden state updates,

\begin{equation*}
    \tilde{\Gamma}(\tilde{\psi}_{t+1}|\tilde{\psi}_t) = \prod_{i \in N} \mu_\omega(s^i_{t+1}) \prod_{i \in N} \pi^i_{\phi_i}(h^i_t),
\end{equation*}
where $\tilde{\psi}_t = (\omega, (s^i_t)_{i \in N}, \mathbf{a}_t, (h^i_t)_{i \in N})$. This gated architecture allows each agent to maintain a compact yet informative representation of their observation history, enabling them to learn sophisticated social learning strategies while keeping the state space tractable.

\subsection*{Centralized Training with Decentralized Execution}
We adopt a CTDE paradigm where each agent maintains a decentralized actor network $\pi^i_{\phi_i}$ for action execution, while training utilizes a centralized critic network $Q_{\xi}$ that has access to the global state. This approach preserves the decentralized nature of action execution while leveraging additional information during training, which mimics the role of a \textit{social planner.}

\subsubsection*{Decentralized Actor Network}
The decentralized actor network for agent $i$ maps the hidden state to a policy distribution as before,
\[
    \pi^i_{\phi_i}(a^i_t|h^i_t) = \text{softmax}(W^i_{\pi}h^i_t + b^i_{\pi}).
\]
Then, we use the policy gradient update to optimize this network given as,
\[
    \nabla_{\phi_i} J(\phi_i) = \mathbb{E}\left[\nabla_{\phi_i} \log \pi^i_{\phi_i}(a^i_t|h^i_t) A^i(\tilde{\psi}_t)\right]
\]
where $A^i(\tilde{\psi}_t)$ is the advantage function computed using the centralized critic.

\subsubsection*{Centralized Critic Network}

The centralized critic takes as input the global state $\psi_t$ and all agents' hidden states:
\[
    Q_{\xi}(\psi_t, \mathbf{h}_t) = W_Q[\psi_t, \mathbf{h}_t, \mathbf{a}_t] + b_Q
\]
where $\mathbf{h}_t = (h^1_t, \ldots, h^n_t)$. The critic network is updated to minimize the \textit{temporal difference} error with the loss function,
\[
    \mathcal{L}(\xi) = \mathbb{E}\left[(r_t + \gamma Q_{\xi}(\tilde{\psi}_{t+1}) - Q_{\xi}(\tilde{\psi}_t))^2\right].
\]
The advantage function for each agent is computed as,
\[
    A^i(\tilde{\psi}_t) = Q_{\xi}(\tilde{\psi}_t) - V_{\xi}(\tilde{\psi}_t),
\]
where $V_{\xi}(\tilde{\psi}_t) = \mathbb{E}_{\mathbf{a}_t \sim \boldsymbol{\pi}}[Q_{\xi}(\tilde{\psi}_t]$ is the value function.


\subsubsection*{Learning Algorithm}
The complete learning process alternates between the actor network update,
\[
    \phi_i \leftarrow \phi_i + \alpha \nabla_{\phi_i} J(\phi_i),
\]
and the critic network update,
\[
    \xi \leftarrow \xi - \beta \nabla_{\xi} \mathcal{L}(\xi).
\]

\subsection*{Single Agent Implementation}

To validate our approach, we first implement a single-agent version of the learning framework. This simplified setting allows us to verify the learning dynamics and compare them against the theoretical bounds established in Brandl (2024). The implementation follows the binary state case ($|\Omega| = 2$) with symmetric signals of accuracy $q > 0.5$.

\subsubsection*{Metrics Tracking}
We track two key metrics throughout the learning process:

\begin{itemize}
    \item \textit{Mistake Rate}: The empirical probability of the agent choosing an incorrect action, computed as the running average of mistakes:
    \[
        P(a_t \neq a_\omega) = \frac{1}{t}\sum_{s=1}^t \mathbf{1}_{\{a_s \neq a_\omega\}}
    \]
    
    \item \textit{Learning Rate}: Following the paper's definition, we compute:
    \[
        r = \liminf_{t \to \infty} -\frac{1}{t} \log P(a_t \neq a_\omega)
    \]
    To approximate this limit, we compute the learning rate over exponentially increasing time windows and take the minimum rate, ensuring we capture the asymptotic behavior.
\end{itemize}

\subsubsection*{Environment}
The environment implements the binary state case with the following components:

\begin{itemize}
    \item A fixed true state $\omega \in \{0,1\}$ drawn uniformly at random at initialization
    \item Signal generation with accuracy $q$:
    \[
        P(s_t = \omega) = q, \quad P(s_t \neq \omega) = 1-q
    \]
    \item The reward function derived in the binary case section:
    \[
        v(a,s) = \frac{q \cdot \mathbf{1}_{\{a = s\}} - (1-q) \cdot \mathbf{1}_{\{a \neq s\}}}{2q-1}
    \]
\end{itemize}

\subsubsection*{Neural Network Architecture}
The agent employs an actor-critic architecture with both networks using GRU cells to process the history of signals. The GRU implementation follows the equations presented in the RNN Architecture section, with the following specifications:

\paragraph{Actor Network}
The actor network maps the signal history to a policy over actions:
\begin{align*}
    h_t &= \text{GRU}(s_t, h_{t-1}) \\
    \pi(a_t|h_t) &= \text{softmax}(W_\pi h_t + b_\pi)
\end{align*}

\paragraph{Critic Network}
The critic network estimates state values using the same GRU architecture:
\begin{align*}
    h_t &= \text{GRU}(s_t, h_{t-1}) \\
    V(h_t) &= W_v h_t + b_v
\end{align*}

\subsubsection*{Training Process}
The agent is trained using a variant of the Advantage Actor-Critic (A2C) algorithm. The complete training procedure is detailed in Algorithm \ref{alg:single-agent-training}.
   
The algorithm implements the A2C training loop, processing signals through GRU-based actor and critic networks while tracking learning metrics. The actor network learns a policy that minimizes mistakes in predicting the true state, while the critic network estimates state values to compute advantages for policy updates.

\subsubsection*{Theoretical Bounds}
For the binary case with symmetric signals, the theoretical bound on the learning rate is:
\[
    r_{bdd} = -\log(1-q_{bdd}) = -\log(1-(2q-1))
\]
where $q_{bdd} = 2q-1$ is the bound on the probability of choosing the correct action. This provides a benchmark against which we can evaluate the empirical learning rate of our implementation.

\subsubsection*{Empirical Results}
The implementation is evaluated over multiple training runs with different random seeds. For each run, we track:
\begin{itemize}
    \item The evolution of the mistake rate over time
    \item The empirical learning rate compared to the theoretical bound
    \item The convergence of the policy to the optimal action
\end{itemize}